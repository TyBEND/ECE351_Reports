%%%%%%%%%%%%%%%%%%%%%%%%%%%%%%%%%%%%%%%%%%%%%%%%%%%%%%%%%%%%%%%%
%                                                              %
% Tyler Bendele                                                %
% ECE351 and Section 51                                        %
% Lab 3                                                        %
% Due Date: February 8, 2022                                   %
% Practice using Discrete Convolution                          %
%                                                              %
%%%%%%%%%%%%%%%%%%%%%%%%%%%%%%%%%%%%%%%%%%%%%%%%%%%%%%%%%%%%%%%%
%%% DOCUMENT PREAMBLE %%%
\documentclass[12pt]{report}
\usepackage[english]{babel}
%\usepackage{natbib}
\usepackage{url}
\usepackage[utf8x]{inputenc}
\usepackage{amsmath}
\usepackage{graphicx}
\graphicspath{{images/}}
\usepackage{parskip}
\usepackage{fancyhdr}
\usepackage{vmargin}
\usepackage{listings}
\usepackage{hyperref}
\usepackage{xcolor}
\usepackage{float}
\definecolor{codegreen}{rgb}{0,0.6,0}
\definecolor{codegray}{rgb}{0.5,0.5,0.5}
\definecolor{codeblue}{rgb}{0,0,0.95}
\definecolor{backcolour}{rgb}{0.95,0.95,0.92}
\lstdefinestyle{mystyle}{
backgroundcolor=\color{backcolour},
commentstyle=\color{codegreen},
keywordstyle=\color{codeblue},
numberstyle=\tiny\color{codegray},
stringstyle=\color{codegreen},
basicstyle=\ttfamily\footnotesize,
breakatwhitespace=false,
breaklines=true,
captionpos=b,
keepspaces=true,
numbers=left,
numbersep=5pt,
showspaces=false,
showstringspaces=false,
showtabs=false,
tabsize=2
}
\lstset{style=mystyle}
\setmarginsrb{3 cm}{2.5 cm}{3 cm}{2.5 cm}{1 cm}{1.5 cm}{1 cm}{1.5 cm}
\title{Lab 3: Discrete Convolution}
% Title
\author{Tyler Bendele}
% Author
\date{February 8, 2022}
% Date
\makeatletter
\let\thetitle\@title
\let\theauthor\@author
\let\thedate\@date
\makeatother
\pagestyle{fancy}
\fancyhf{}
\rhead{\theauthor}
\lhead{\thetitle}
\cfoot{\thepage}
%%%%%%%%%%%%%%%%%%%%%%%%%%%%%%%%%%%%%%%%%%%%
\begin{document}
%%%%%%%%%%%%%%%%%%%%%%%%%%%%%%%%%%%%%%%%%%%%%%%%%%%%%%%%%%%%%%%%%%%%%%%%%%
%%%%%%%%%%%%%%%
\begin{titlepage}
\centering
\vspace*{0.5 cm}
% \includegraphics[scale = 0.075]{bsulogo.png}\\[1.0 cm] % 

\begin{center}    \textsc{\Large   ECE 351 - Section \#51 }\\[2.0 cm]
\end{center}% University Name
\textsc{\Large Signals and Systems  }\\[0.5 cm] % Course 

\rule{\linewidth}{0.2 mm} \\[0.4 cm]
{ \huge \bfseries \thetitle}\\
\rule{\linewidth}{0.2 mm} \\[1.5 cm]
\begin{minipage}{0.4\textwidth}
\begin{flushleft} \large
% \emph{Submitted To:}\\
% Name\\
% Affiliation\\
%contact info\\
\end{flushleft}
\end{minipage}~
\begin{minipage}{0.4\textwidth}
\begin{flushright} \large
\emph{Submitted By :} \\
Tyler Bendele
\end{flushright}
\end{minipage}\\[2 cm]
% \includegraphics[scale = 0.5]{PICMathLogo.png}
\end{titlepage}
%%%%%%%%%%%%%%%%%%%%%%%%%%%%%%%%%%%%%%%%%%%%%%%%%%%%%%%%%%%%%%%%%%%%%%%%%%
%%%%%%%%%%%%%%%
\tableofcontents
\pagebreak
%%%%%%%%%%%%%%%%%%%%%%%%%%%%%%%%%%%%%%%%%%%%%%%%%%%%%%%%%%%%%%%%%%%%%%%%%%
%%%%%%%%%%%%%%%
\renewcommand{\thesection}{\arabic{section}}
\section{Introduction}
The goal of this lab is to become familiar with convolution and its 
Properties using Python. A convolution is used to combine two signals 
into one.

\section{Equations}
The equations shown below includes the 3 functions made in Part 1 and then
used throughout the rest of the lab. The last function is the convolution
integral.
\begin{equation}
    f_{1}(t) = u(t-2) - u(t-9)
\end{equation}
\begin{equation}
    f_{2}(t) = e^{-t}u(t)
\end{equation}
\begin{equation}
    f_{3}(t) = r(t-2)[u(t-2) - u(t-3)] + r(4-t)[u(t-3) - u(t-4)]
\end{equation}
\begin{equation}
    \int_{0}^{t} h(\tau)\cdot f(t-\tau) d\tau 
\end{equation}
\section{Methodology}
\subsection{Part 1}
In this first part of the lab, I was able to apply what I learned from
Lab 2, to create the user defined functions shown in the first 3 equations.
The code to create each function, as well as the graph holding all 3
functions is shown down below.

\begin{lstlisting}[language=Python]
steps = 1e-2
t = np.arange(0, 20 + steps, steps)

def u(t):          #step function
    y = np.zeros(t.shape)
    for i in range(len(t)):
        if t[i] > 0:
            y[i] = 1
        else:
            y[i] = 0
    return y

def r(t):          #ramp function
    y = np.zeros(t.shape)
    for i in range(len(t)):
       if t[i] > 0:
           y[i] = t[i]
       else:
           y[i] = 0
    return y

def f1(t):
    a = np.zeros(t.shape)
    a = u(t-2) - u(t-9)
    return a
a = f1(t)

def f2(t):
    b = np.zeros(t.shape)
    b = np.exp(-t)*u(t)
    return b
b = f2(t)

def f3(t):
    c = np.zeros(t.shape)
    c = r(t-2) * (u(t-2) - u(t-3)) + r(4-t) * (u(t-3) - u(t-4))
    return c
c = f3(t)

plt.figure(figsize=(12,8))
plt.subplot(3,1,1)
plt.plot(t,a)
plt.title('Part 1: User Defined Functions') 
                                 
plt.ylabel('f1(t)')
plt.grid(True)

plt.subplot(3,1,2)
plt.plot(t,b)
plt.ylabel('f2(t)') 
plt.grid(which='both')

plt.subplot(3,1,3)
plt.plot(t,c)
plt.grid(True)
plt.xlabel('t') 
plt.ylabel('f3(t)') 
plt.show() 
\end{lstlisting}
\newpage
\subsection{Part 2}
In this section, part 2 task 1, we had to create the code that would 
perform the convolution between two functions. I definitely needed help
making the code for convolution. In this method, the convolution
function takes in two functions and immediately finds the length of them.
An extended array is made for each function using the other. The return value is then created by filling an array, the size of the first function,
full of zeros. The array then gets filled up with the combined values of
both functions using a nested for loop. The outside loop goes for the
length of both function minus 2. The inside loop goes for the length
of the first function. In the inside loop, the extended arrays are
multiplied together.
\begin{lstlisting}[language=Python]
def my_conv(f1,f2):
    Nf1 = len(f1)
    Nf2 = len(f2)
    f1Ext = np.append(f1, np.zeros((1, Nf2 - 1)))
    f2Ext = np.append(f2, np.zeros((1, Nf1 - 1)))
    result = np.zeros(f1Ext.shape)
    for i in range(Nf2 + Nf1 - 2):
        result[i] = 0
        for j in range(Nf1):
            if ((i - j + 1) > 0):
                try: 
                    result[i] += f1Ext[j] * f2Ext[i - j + 1]
                except:
                    print(i, j)
    return result
\end{lstlisting}
Tasks 2, 3, and 4 involved using the convolution function that we created in part 
2 to convolve different functions from part 1. Specifically convolving: functions
1 and 2 for task 2, functions 1 and 3 for task 3, and functions 
2 and 3 for task 4. These were graphed just like in the last lab,
except the range of t had to be modified to be twice the size.

\begin{lstlisting}[language=Python]
t = np.arange(0, 20 + steps, steps)
N = len(t)
tExt = np.arange(0, 2 * t[N-1], steps)

x = my_conv(a, b) * steps

plt.figure(figsize = (10, 7))
plt.subplot(2, 1, 1)
plt.plot(tExt, x)
plt.grid()
plt.ylabel('x(t)')
plt.xlabel('t')
plt.title('Convolution of f1 and f2')
plt.show()

y = my_conv(a, c) * steps

plt.figure(figsize = (10, 7))
plt.subplot(2, 1, 1)
plt.plot(tExt, y)
plt.grid()
plt.ylabel('y(t)')
plt.xlabel('t')
plt.title('Convolution of f1 and f3')
plt.show()

z = my_conv(b, c) * steps

plt.figure(figsize = (10, 7))
plt.subplot(2, 1, 1)
plt.plot(tExt, z)
plt.grid()
plt.ylabel('z(t)')
plt.xlabel('t')
plt.title('Convolution of f2 and f3')
plt.show()
\end{lstlisting}
\newpage
\section{Results}
Looking at the different graphs for the lab, they all seem to have graphed
correctly based on the function inputted.
\subsection{Part 1}
\begin{figure}[H]
\begin{center}
\caption{Part 1, Task 2}
\includegraphics[scale=0.55]{L3_part1_task2.png}
\end{center}
\end{figure}

\subsection{Part 2} {
\begin{figure}[H]
\begin{center}
\caption{Part 2, Task 2, Convolution of Functions 1 and 2}
\includegraphics[scale=0.6]{L3_part2_task2.png}
\end{center}
\end{figure}

\begin{figure}[H]
\begin{center}
\caption{Part 2, Task 3, Convolution of Functions 1 and 3}
\includegraphics[scale=0.6]{L3_part2_task3.png}
\end{center}
\end{figure}

\begin{figure}[H]
\begin{center}
\caption{Part 2, Task 4, Convolution of Functions 2 and 3}
\includegraphics[scale=0.6]{L3_part2_task4.png}
\end{center}
\end{figure}
}

\section{Error Analysis}
Based on the graphs all printing correctly, I believe there weren't
any errors in my code. I don't think I had too many problems
when doing this lab. For the most part it went pretty smoothly.
I did have a couple of errors trying to get it to run after
adding in the convolution function. However, I figured out that
the problem had to do with how I was printing it.

I did however find it difficult to completely understand how
exactly the convolution integral worked. I think I mostly 
understand it.

\section{Questions}
1. Did you work alone or with classmates on this lab? If you collaborated to get to the solution, what did that process look like?

No, I didn't work alone for the convolution part of the lab. I found it 
pretty difficult to come up with the solution on my own.The collaboration
actually ended up involving everyone in the lab. We all worked on the
lab separately for the for the first hour. Then for the second hour, we
went over the solution line by line with the TA explaining how each part
worked.

2. What was the most difficult part of this lab for you, and what did your
problem-solving process look like?

The most difficult part of the lab, definitely involved coming up with 
the convolution function. Since I had quite a bit of help on that part,
my problem-solving process mostly involved checking to see if it ran
correctly, and if it didn't I would check for typos. 

3. Did you approach writing the code with analytical or graphical 
convolution in mind? Why did you chose this approach?

When I was approaching the problem, I thought about the code through
graphical convolution. For me this is the easiest way to think about it.
I also think it is difficult to think about it analytically without
actually taking the integral.

4. Leave any feedback on the clarity of lab tasks, expectations, and 
deliverables.

I think this lab was very clear on the lab tasks, expectations, and 
deliverables, so I don't really have any feedback.

\section{Conclusion}
In this lab, I learned more about the convolution integral, and how to use
it in Python. Specifically, how to find the graphical convolution between
two different functions. I also learned how it can be implemented, through 
graphing as well. The lab was pretty clear in what it was expecting.
Convolution is a difficult concept to understand at first, so I don't know
if there is an easy way to implement learning it easier in the lab.
However, I think the lab is great at assisting in learning how convolution
works conceptually because you can use the convolution function on 
any two functions and it will visually show what they look convolved
together. I think learning the convolution function will be a helpful tool
for future labs.
\end{document}
%This template was created by Roza Aceska.