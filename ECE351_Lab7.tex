%%%%%%%%%%%%%%%%%%%%%%%%%%%%%%%%%%%%%%%%%%%%%%%%%%%%%%%%%%%%%%%%
%                                                              %
% Tyler Bendele                                                %
% ECE351 and Section 51                                        %
% Lab 7                                                        %
% Due Date: March 8, 2022                                      %
% Block Diagram and System Stability                           %
%                                                              %
%%%%%%%%%%%%%%%%%%%%%%%%%%%%%%%%%%%%%%%%%%%%%%%%%%%%%%%%%%%%%%%%
%%% DOCUMENT PREAMBLE %%%
\documentclass[12pt]{report}
\usepackage[english]{babel}
%\usepackage{natbib}
\usepackage{url}
\usepackage[utf8x]{inputenc}
\usepackage{amsmath}
\usepackage{graphicx}
\graphicspath{{images/}}
\usepackage{parskip}
\usepackage{fancyhdr}
\usepackage{vmargin}
\usepackage{listings}
\usepackage{hyperref}
\usepackage{xcolor}
\usepackage{float}
\definecolor{codegreen}{rgb}{0,0.6,0}
\definecolor{codegray}{rgb}{0.5,0.5,0.5}
\definecolor{codeblue}{rgb}{0,0,0.95}
\definecolor{backcolour}{rgb}{0.95,0.95,0.92}
\lstdefinestyle{mystyle}{
backgroundcolor=\color{backcolour},
commentstyle=\color{codegreen},
keywordstyle=\color{codeblue},
numberstyle=\tiny\color{codegray},
stringstyle=\color{codegreen},
basicstyle=\ttfamily\footnotesize,
breakatwhitespace=false,
breaklines=true,
captionpos=b,
keepspaces=true,
numbers=left,
numbersep=5pt,
showspaces=false,
showstringspaces=false,
showtabs=false,
tabsize=2
}
\lstset{style=mystyle}
\setmarginsrb{3 cm}{2.5 cm}{3 cm}{2.5 cm}{1 cm}{1.5 cm}{1 cm}{1.5 cm}
\title{Lab 7: Block Diagrams and System Stability}
% Title
\author{Tyler Bendele}
% Author
\date{March 1, 2022}
% Date
\makeatletter
\let\thetitle\@title
\let\theauthor\@author
\let\thedate\@date
\makeatother
\pagestyle{fancy}
\fancyhf{}
\rhead{\theauthor}
\lhead{\thetitle}
\cfoot{\thepage}
%%%%%%%%%%%%%%%%%%%%%%%%%%%%%%%%%%%%%%%%%%%%
\begin{document}
%%%%%%%%%%%%%%%%%%%%%%%%%%%%%%%%%%%%%%%%%%%%%%%%%%%%%%%%%%%%%%%%%%%%%%%%%%
%%%%%%%%%%%%%%%
\begin{titlepage}
\centering
\vspace*{0.5 cm}
% \includegraphics[scale = 0.075]{bsulogo.png}\\[1.0 cm] % 

\begin{center}    \textsc{\Large   ECE 351 - Section \#51 }\\[2.0 cm]
\end{center}% University Name
\textsc{\Large Signals and Systems  }\\[0.5 cm] % Course 

\rule{\linewidth}{0.2 mm} \\[0.4 cm]
{ \huge \bfseries \thetitle}\\
\rule{\linewidth}{0.2 mm} \\[1.5 cm]
\begin{minipage}{0.4\textwidth}
\begin{flushleft} \large
% \emph{Submitted To:}\\
% Name\\
% Affiliation\\
%contact info\\
\end{flushleft}
\end{minipage}~
\begin{minipage}{0.4\textwidth}
\begin{flushright} \large
\emph{Submitted By :} \\
Tyler Bendele
\end{flushright}
\end{minipage}\\[2 cm]
% \includegraphics[scale = 0.5]{PICMathLogo.png}
\end{titlepage}
%%%%%%%%%%%%%%%%%%%%%%%%%%%%%%%%%%%%%%%%%%%%%%%%%%%%%%%%%%%%%%%%%%%%%%%%%%
%%%%%%%%%%%%%%%
\tableofcontents
\pagebreak
%%%%%%%%%%%%%%%%%%%%%%%%%%%%%%%%%%%%%%%%%%%%%%%%%%%%%%%%%%%%%%%%%%%%%%%%%%
%%%%%%%%%%%%%%%
\renewcommand{\thesection}{\arabic{section}}
\section{Introduction}
The purpose of this lab is to analyze block diagrams by hand and learn how
to use python functions to perform the same analysis on those diagrams.
In this lab we will also discuss the stability of the block diagram
for both the open and closed responses.
\section{Equations}
In the first three equations, I break up the given equations for G, A, and 
B into their roots and poles. The fourth equation shows the equation from part
1 task 3, where I find the the open loop transfer function.
The fifth equation shows the closed loop transfer function split up into 
the parts of the first 3 equations based on their numerators and
denominators. The sixth and final equation show the closed loop response
broken down into its poles and roots that python calculates using the
scipy.signal.tf2zpk() function.
\begin{equation}
    G(s) = \frac{s + 9}{(s - 8)(s + 2)(2 + 4)}
\end{equation}
\begin{equation}
    A(s) = \frac{s+4}{(s+3)(s+1)}
\end{equation}
\begin{equation}
    B(s) = (s+12)(s+14)
\end{equation}
\begin{equation}
    \frac{Y_{Open}(s)}{X(s)} = \frac{(s+9)(s+4)}{(s-8)(s+2)(s+4)(s+3)(s+1)}
\end{equation}
\begin{equation}
    \frac{Y_{Closed}(s)}{X(s)} = \frac{(numA)(numG)}{(denA)(denG) + (denA)(numB)(numG)}
\end{equation}
\begin{equation}
    \frac{Y_{Closed}(s)}{X(s)} = \frac{(s+9)(s+4)}{(s+5.162-9.518)(s+5.162+9.518)(s+6.175)(s+3)(s+1)}
\end{equation}
\subsection{Part 1}
For the first task, I just separated the functions of G, B, and A into
their poles and roots. These equations can also be seen in the first
3 equations in the equation section. For the second task I practiced using
the scipy.signal.tf2zpk() and numpy.roots() functions to find the poles
and roots of the functions using python. The code for this can be seen down
below.

\begin{lstlisting}[language=Python]
num = [0, 0, 1, 9] # Creates a matrix for the numerator
den = [1, -2, -40, -64] # Creates a matrix for the denominator

z, p, k= sig.tf2zpk(num, den)
print(z)
print(p)
print(k)

num2 = [0, 1, 4] # Creates a matrix for the numerator
den2 = [1, 4, 3] # Creates a matrix for the denominator

z, p, k= sig.tf2zpk(num2, den2)
print(z)
print(p)
print(k)

num3 = [1, 26, 168] # Creates a matrix for the numerator

b = np.roots(num3)
print(b)
\end{lstlisting}
The third task involved writing the open loop equation of the block diagram
in factored form, which can be seen in equation number 4 of the equation
section. Task 4 asks if the open loop equation is stable.
In that equation it can be seen that the open loop equation isn't stable
because it has a positive root. For the fifth task, I find the step
response of the open loop response and graph it. Task 6, asks if the
step response supports my answer for task 4. It does in fact support my
answer as the plot ends up going off towards infinity showing that it isn't
stable.
\begin{lstlisting}[language=Python]
num4 = sig.convolve([1,9] , [1,4])
print('Numerator =', num4)

den4 = sig.convolve([1, -2, -40, -64] , [1, 4, 3])
print('Denominator =', den4)

tout, yout = sig.step((num4, den4), T=t)

plt.figure(figsize = (10, 7))
plt.plot(tout, yout)
plt.grid()
plt.ylabel('Step response')
plt.xlabel('t')
plt.title('Open Loop Step Response')
plt.show()
\end{lstlisting}
\subsection{Part 2}
For the first task in part 2, I show the result from splitting up the
closed loop response into the components of the numerators and denominators
of the original three equations given. This equation can be seen as the 
fifth equation in the equation section. For the second task I convolved the
different numerators and denominators to form that function using the
convolve scipy function. I then used the tf2zpk function to find the
poles and roots of the closed loop transfer function. The code for this can
be seen down below. The equation that results from this, can be seen as the
sixth equation.
\begin{lstlisting}[language=Python]
numA = [1, 4]
numG = [1, 9]
numB = [1, 26, 168]
denA = [1, 4, 3]
denG = [1, -2, -40, -64]

num5 = sig.convolve(numA, numG)
print(num5)

den5 = sig.convolve(denA, denG)
print(den5)

num6 = sig.convolve(numB, numG)
den6 = sig.convolve(num6, denA)
print(den6)

den7 = den5 + den6
print(den7)

z, p, k= sig.tf2zpk(num5, den7)
print(z)
print(p)
print(k)
\end{lstlisting}
The third task asks if the closed-loop transfer function is stable.
As seen in the sixth equation, there aren't any real negative roots so
the closed-loop response transfer function is stable. In the fourth
task I find the step response of the closed-loop equation. The plot looks
like it begins to level off as it goes towards infinity, so it does support
the claim that it is stable.
\begin{lstlisting}[language=Python]
tout2, yout2 = sig.step((num5, den7), T=t)

plt.figure(figsize = (10, 7))
plt.plot(tout2, yout2)
plt.grid()
plt.ylabel('Step response')
plt.xlabel('t')
plt.title('Closed Loop Step Response')
plt.show()
\end{lstlisting}
\section{Results}
\subsection{Part 1}
\begin{figure}[H]
\begin{center}
\caption{Part 1, Tasks 2, Equation G Expanded}
\includegraphics[scale=0.80]{eqGroot.png}
\end{center}
\end{figure}

\begin{figure}[H]
\begin{center}
\caption{Part 1, Tasks 2, Equation A Expanded}
\includegraphics[scale=0.80]{eqAroot.png}
\end{center}
\end{figure}

\begin{figure}[H]
\begin{center}
\caption{Part 1, Tasks 2, Equation B Expanded}
\includegraphics[scale=0.80]{eqBroot.png}
\end{center}
\end{figure}

\begin{figure}[H]
\begin{center}
\caption{Part 1, Tasks 4, Open Loop Convolve Output}
\includegraphics[scale=0.80]{openloopOutput.png}
\end{center}
\end{figure}

\begin{figure}[H]
\begin{center}
\caption{Part 1, Task 4, Open Loop Step Response}
\includegraphics[scale=0.50]{openloop.png}
\end{center}
\end{figure}

\subsection{Part 2}
\begin{figure}[H]
\begin{center}
\caption{Part 2, Task 2, Closed Loop Output}
\includegraphics[scale=0.80]{pt2task2.png}
\end{center}
\end{figure}

\begin{figure}[H]
\begin{center}
\caption{Part 2, Tasks 4, Closed Loop Step Response}
\includegraphics[scale=0.50]{closedloop.png}
\end{center}
\end{figure}


\section{Error Analysis}
I did not have any errors occur this lab. I did however have confusion
starting the lab, because I didn't know there was a difference
between the closed and open loop transfer functions. 

\section{Questions}
1. In Part 1 Task 5, why does convolving the factored terms using scipy.signal.convolve()
result in the expanded form of the numerator and denominator? Would this work with your
user-defined convolution function from Lab 3? Why or why not?

Convolving the factored terms combines the two functions, which is exactly what happens
when you multiply two factored terms together, they combine to create one function or
polynomial. This would not work on the the user-defined functions from lab 3, because
those equations were created using the step and ramp functions which multiply and combine
together differently.

2. Discuss the difference between the open- and closed-loop systems from Part 1 and Part 
2. How does stability differ for each case, and why?

The open loop takes the most direct route through the block diagram, and only includes
those terms in its response transfer equation. The closed loop system on the other hand
takes into account everything in the block diagram including the part that loops back on
to itself. Generally, closed loop systems are used to make the system more stable
otherwise open loop systems would be used more often since they are faster. In the
specific case of this lab, the closed loop system ended up being stable whereas the
the open loop part wasn't. I believe closed loop systems are more stable because
the system is able to loop back on to itself which allows it to reprocess problems
that otherwise would have risen only going through the system once.

3. What is the difference between scipy.signal.residue() used in Lab 6 and
scipy.signal.tf2zpk() used in this lab?

The scipy.signal.tf2zpk() function finds the poles and roots of the transfer function,
which is useful in confirming the denominators found when performing the
scipy.signal.residue() function.
The scipy.signal.residue() performs partial fraction expansion. This allows us to find
the step response equation.

4. Is it possible for an open-loop system to be stable? What about for a closed-loop 
system to be unstable? Explain how or how not for each.

I think both are definitely possible. Whether a system is stable or unstable is more
dependent on the different subsystems that are used. Since, the closed-loop system usually
involves a feedback path, it is more likely to be stable because of how the equation for
a feedback is setup.

5. Leave any feedback on the clarity/usefulness of the purpose, deliverables, and 
expectations for this lab.

I think everything the deliverables, purpose, and expectations were pretty clear for the
lab. The only thing I had confusion on was that there was a difference between closed
loop and open loop systems, which was cleared up.

\section{Conclusion}
I believe this lab was pretty useful in teaching methods for analyzing block diagrams.
I was able to learn about the difference between open and closed loop systems and how
their stability compares. I also learned methods for analyzing transfer functions by
breaking them up into factored form.

\end{document}
%This template was created by Roza Aceska.