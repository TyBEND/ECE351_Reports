%%%%%%%%%%%%%%%%%%%%%%%%%%%%%%%%%%%%%%%%%%%%%%%%%%%%%%%%%%%%%%%%
%                                                              %
% Tyler Bendele                                                %
% ECE351 and Section 51                                        %
% Lab 6                                                        %
% Due Date: March 1, 2022                                      %
% Partial Fraction Expansion practice                          %
%                                                              %
%%%%%%%%%%%%%%%%%%%%%%%%%%%%%%%%%%%%%%%%%%%%%%%%%%%%%%%%%%%%%%%%
%%% DOCUMENT PREAMBLE %%%
\documentclass[12pt]{report}
\usepackage[english]{babel}
%\usepackage{natbib}
\usepackage{url}
\usepackage[utf8x]{inputenc}
\usepackage{amsmath}
\usepackage{graphicx}
\graphicspath{{images/}}
\usepackage{parskip}
\usepackage{fancyhdr}
\usepackage{vmargin}
\usepackage{listings}
\usepackage{hyperref}
\usepackage{xcolor}
\usepackage{float}
\definecolor{codegreen}{rgb}{0,0.6,0}
\definecolor{codegray}{rgb}{0.5,0.5,0.5}
\definecolor{codeblue}{rgb}{0,0,0.95}
\definecolor{backcolour}{rgb}{0.95,0.95,0.92}
\lstdefinestyle{mystyle}{
backgroundcolor=\color{backcolour},
commentstyle=\color{codegreen},
keywordstyle=\color{codeblue},
numberstyle=\tiny\color{codegray},
stringstyle=\color{codegreen},
basicstyle=\ttfamily\footnotesize,
breakatwhitespace=false,
breaklines=true,
captionpos=b,
keepspaces=true,
numbers=left,
numbersep=5pt,
showspaces=false,
showstringspaces=false,
showtabs=false,
tabsize=2
}
\lstset{style=mystyle}
\setmarginsrb{3 cm}{2.5 cm}{3 cm}{2.5 cm}{1 cm}{1.5 cm}{1 cm}{1.5 cm}
\title{Lab 6: Partial Fraction Expansion}
% Title
\author{Tyler Bendele}
% Author
\date{March 1, 2022}
% Date
\makeatletter
\let\thetitle\@title
\let\theauthor\@author
\let\thedate\@date
\makeatother
\pagestyle{fancy}
\fancyhf{}
\rhead{\theauthor}
\lhead{\thetitle}
\cfoot{\thepage}
%%%%%%%%%%%%%%%%%%%%%%%%%%%%%%%%%%%%%%%%%%%%
\begin{document}
%%%%%%%%%%%%%%%%%%%%%%%%%%%%%%%%%%%%%%%%%%%%%%%%%%%%%%%%%%%%%%%%%%%%%%%%%%
%%%%%%%%%%%%%%%
\begin{titlepage}
\centering
\vspace*{0.5 cm}
% \includegraphics[scale = 0.075]{bsulogo.png}\\[1.0 cm] % 

\begin{center}    \textsc{\Large   ECE 351 - Section \#51 }\\[2.0 cm]
\end{center}% University Name
\textsc{\Large Signals and Systems  }\\[0.5 cm] % Course 

\rule{\linewidth}{0.2 mm} \\[0.4 cm]
{ \huge \bfseries \thetitle}\\
\rule{\linewidth}{0.2 mm} \\[1.5 cm]
\begin{minipage}{0.4\textwidth}
\begin{flushleft} \large
% \emph{Submitted To:}\\
% Name\\
% Affiliation\\
%contact info\\
\end{flushleft}
\end{minipage}~
\begin{minipage}{0.4\textwidth}
\begin{flushright} \large
\emph{Submitted By :} \\
Tyler Bendele
\end{flushright}
\end{minipage}\\[2 cm]
% \includegraphics[scale = 0.5]{PICMathLogo.png}
\end{titlepage}
%%%%%%%%%%%%%%%%%%%%%%%%%%%%%%%%%%%%%%%%%%%%%%%%%%%%%%%%%%%%%%%%%%%%%%%%%%
%%%%%%%%%%%%%%%
\tableofcontents
\pagebreak
%%%%%%%%%%%%%%%%%%%%%%%%%%%%%%%%%%%%%%%%%%%%%%%%%%%%%%%%%%%%%%%%%%%%%%%%%%
%%%%%%%%%%%%%%%
\renewcommand{\thesection}{\arabic{section}}
\section{Introduction}
The purpose of this lab is to practice using partial fraction expansion
using the function scipy.signal.residue().
\section{Equations}
The first two equations used in lab, that are shown below, were found during the 
prelab. The first is the transfer function and the second is the 
step response that was found. The third equation is transfer function of the
complicated equation given for part 2 of the lab. Finally, the fourth equation
is the cosine method.
\begin{equation}
    H(s) = \frac{s^{2} + 6s + 12}{s^{3} + 10s^{2} + 24s}
\end{equation}
\begin{equation}
    y(t) = (\frac{1}{2} - \frac{1}{2}e^{-4t} + e^{-6t}) \cdot u(t)
\end{equation}
\begin{equation}
    H(s) = \frac{25250}{s^{6} + 18s^{5} + 218s^{4} + 2036s^{3} + 9085s^{2} + 25250s}
\end{equation}
\begin{equation}
    y_{c} = 2|k|e^{\alpha t}cos(\omega t +	\angle k) \cdot u(t)
\end{equation}

\section{Methodology}
\subsection{Part 1}
For the first part, we graphed the partial fraction expansion equation
that we found in the prelab. We then found the step response of the first 
equation and compared it on the same graph. This was done using functions
we learned from previous labs. 

\begin{lstlisting}[language=Python]
def u(t):          #step function
    y = np.zeros(t.shape)
    for i in range(len(t)):
        if t[i] > 0:
            y[i] = 1
        else:
            y[i] = 0
    return y

def impulse(t):
    y = (0.5 - 0.5*np.exp(-4*t) + np.exp(-6*t))*u(t)
    return y


num = [1, 6, 12] # Creates a matrix for the numerator
den = [1, 10, 24] # Creates a matrix for the denominator

tout , yout = sig.step(( num , den) , T = t )

# Plot tout , yout
plt.figure(figsize = (10, 7))
plt.subplot(2, 1, 1)
plt.plot(t, impulse(t))
plt.grid()
plt.ylabel('Hand Calculated')
plt.title('Impulse Response')

plt.subplot(2, 1, 2)
plt.plot(tout, yout)
plt.grid()
plt.ylabel('Scipy Impulse Solution') 
plt.xlabel('t')
plt.show()
\end{lstlisting}
For the third task we used the scipy.signal.residue() function on the 
transfer function that we found, to find the partial fraction 
expansion. The code for this was done using arrays to make up the
numerator and denominator and then printed using the different variables
of the residue function.

\begin{lstlisting}[language=Python]
num2 = [0, 1, 6, 12] # Creates a matrix for the numerator
den2 = [1, 10, 24, 0] # Creates a matrix for the denominator

R, P, K = sig.residue(num2, den2)

print(R)
print(P)
print(K)
\end{lstlisting}
\subsection{Part 2}
In this section we practiced using the residue function on an equation
that would be difficult to use partial fraction expansion on by hand. The
transfer function of this equation can be seen in equation 3.
\begin{lstlisting}[language=Python]
num3 = [0, 0, 0, 0, 0, 0, 25250] # Creates a matrix for the numerator
den3 = [1, 18, 218, 2036, 9085, 25250, 0] # Creates a matrix for the denominator

R, P, K = sig.residue(num3, den3)

print(R)
print(P)
print(K)
\end{lstlisting}
For the second task, we needed to use the cosine method for the results
from the partial fraction expansion. The equation for the cosine method
can be seen in equation 4. I did this by again creating an
array for the numerator and the denominator of the fractions of the results
from the residue function. However I specifically only did this with the
mixed number results from the function. I then put these arrays into a 
user defined function that inputted each entry from the arrays into its
separate cosine method equation. This was done using a for loop to go
through each entry in the arrays and add it on to the previous result in the
array. Finally I combined all of these cosine functions found with the two
results that didn't involve imaginary numbers to get the final equation which was then graphed. In the third task of part 2, the third equation was
graphed again, using the step response this time. These two graphs, which
should be identical, were put on the same figure for comparison.
\begin{lstlisting}[language=Python]
resn1 = [- 0.48557692+0.72836538j, -0.48557692-0.72836538j,
        0.09288674-0.04765193j, 0.09288674+0.04765193j]

resd1 = [-3 + 4.j,  -3. -4.j, -1.+10.j, -1.-10.j]

def cosmethod(d, n):
    y = 0
    for i in range(len(d)):
        k = np.abs(n[i])
        ka = np.angle(n[i])
        a = np.real(d[i])
        o = np.imag(d[i])
        y += k*np.exp(a*t)*np.cos(o*t + ka)
    return y

t = np.arange(0, 4.5 + steps, steps)

y = (cosmethod(resd1, resn1) + 1 - 0.21461963*np.exp(-10*t))*u(t)

# Part 2 Task 3
den4 = [1, 18, 218, 2036, 9085, 25250]
tout2, yout2 = sig.step((num3, den4), T = t)

plt.figure(figsize = (10, 7))
plt.subplot(2, 1, 1)
plt.plot(t, y)
plt.grid()
plt.ylabel ('hand solved')

plt.subplot (2, 1, 2)
plt.plot (tout2, yout2)
plt.grid()
plt.ylabel('scipy solved')
plt.xlabel('t')
plt.show()
\end{lstlisting}
\section{Results}

\subsection{Part 1}
\begin{figure}[H]
\begin{center}
\caption{Part 1, Tasks 1 \& 2, Impulse Response}
\includegraphics[scale=0.50]{part1graphes.png}
\end{center}
\end{figure}

\begin{figure}[H]
\begin{center}
\caption{Part 1, Tasks 3, Printed Partial Fraction Expansion Results}
\includegraphics[scale=0.80]{Part1printedresult.png}
\end{center}
\end{figure}

\subsection{Part 2}
\begin{figure}[H]
\begin{center}
\caption{Part 2, Task 1, Printed PFE of equation 3}
\includegraphics[scale=0.80]{Part2printedresult.png}
\end{center}
\end{figure}

\begin{figure}[H]
\begin{center}
\caption{Part 2, Tasks 2 \& 3}
\includegraphics[scale=0.50]{part2graphes.png}
\end{center}
\end{figure}


\section{Error Analysis}
I don't believe I had too many errors this time around. The only problems I had were getting the
step response function to work and figuring out the syntax for the residue function. I was able to
fix the first thing by changing the size of the steps. As for the syntax of the residue function,
I was eventually able to find the syntax through a little bit of researching
and checking.

\section{Questions}
1. For a non-complex pole residue term, you can still use the cosine
method, explain why this works.

It works because the imaginary part and the angle from the mixed number
make up the the input to the cosine function in the cosine method. However,
this input to the cosine function is zero since the imaginary part is zero
in the case of a non-complex pole and the angle is also zero. This makes
the cosine part equal to 1. The rest of the cosine method is based on the
magnitude and real part, which end up making up the normal result of a
non-complex residue term.

2. Leave any feedback on the clarity of the expectations, instructions, and deliverables.

It seems like everything in this lab was pretty clear on the clarity of the
expectations, instructions, and deliverables. 

\section{Conclusion}
This lab was very useful because we learned how to use the residue
function. This knowledge allows us to find the partial fraction expansion
of more complicated signals that otherwise would have been impossible to
do by hand. It also helped me practice in using the cosine method to find the
step response of complicated complex pole-residue terms.

\end{document}
%This template was created by Roza Aceska.