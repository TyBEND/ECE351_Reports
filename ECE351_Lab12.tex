%%%%%%%%%%%%%%%%%%%%%%%%%%%%%%%%%%%%%%%%%%%%%%%%%%%%%%%%%%%%%%%%
%                                                              %
% Tyler Bendele                                                %
% ECE351 and Section 51                                        %
% Lab 12 Filter Design                                         %
% Due Date: April 26, 2022                                     %
% Final Project                                                %
%                                                              %
%%%%%%%%%%%%%%%%%%%%%%%%%%%%%%%%%%%%%%%%%%%%%%%%%%%%%%%%%%%%%%%%
%%% DOCUMENT PREAMBLE %%%
\documentclass[12pt]{report}
\usepackage[english]{babel}
%\usepackage{natbib}
\usepackage{url}
\usepackage[utf8x]{inputenc}
\usepackage{amsmath}
\usepackage{graphicx}
\graphicspath{{images/}}
\usepackage{parskip}
\usepackage{fancyhdr}
\usepackage{vmargin}
\usepackage{listings}
\usepackage{hyperref}
\usepackage{xcolor}
\usepackage{float}
\definecolor{codegreen}{rgb}{0,0.6,0}
\definecolor{codegray}{rgb}{0.5,0.5,0.5}
\definecolor{codeblue}{rgb}{0,0,0.95}
\definecolor{backcolour}{rgb}{0.95,0.95,0.92}
\lstdefinestyle{mystyle}{
backgroundcolor=\color{backcolour},
commentstyle=\color{codegreen},
keywordstyle=\color{codeblue},
numberstyle=\tiny\color{codegray},
stringstyle=\color{codegreen},
basicstyle=\ttfamily\footnotesize,
breakatwhitespace=false,
breaklines=true,
captionpos=b,
keepspaces=true,
numbers=left,
numbersep=5pt,
showspaces=false,
showstringspaces=false,
showtabs=false,
tabsize=2
}
\lstset{style=mystyle}
\setmarginsrb{3 cm}{2.5 cm}{3 cm}{2.5 cm}{1 cm}{1.5 cm}{1 cm}{1.5 cm}
\title{Lab 12: Filter Design}
% Title
\author{Tyler Bendele}
% Author
\date{April 26, 2022}
% Date
\makeatletter
\let\thetitle\@title
\let\theauthor\@author
\let\thedate\@date
\makeatother
\pagestyle{fancy}
\fancyhf{}
\rhead{\theauthor}
\lhead{\thetitle}
\cfoot{\thepage}
%%%%%%%%%%%%%%%%%%%%%%%%%%%%%%%%%%%%%%%%%%%%
\begin{document}
%%%%%%%%%%%%%%%%%%%%%%%%%%%%%%%%%%%%%%%%%%%%%%%%%%%%%%%%%%%%%%%%%%%%%%%%%%
%%%%%%%%%%%%%%%
\begin{titlepage}
\centering
\vspace*{0.5 cm}
% \includegraphics[scale = 0.075]{bsulogo.png}\\[1.0 cm] % 

\begin{center}    \textsc{\Large   ECE 351 - Section \#51 }\\[2.0 cm]
\end{center}% University Name
\textsc{\Large Signals and Systems  }\\[0.5 cm] % Course 

\rule{\linewidth}{0.2 mm} \\[0.4 cm]
{ \huge \bfseries \thetitle}\\
\rule{\linewidth}{0.2 mm} \\[1.5 cm]
\begin{minipage}{0.4\textwidth}
\begin{flushleft} \large
% \emph{Submitted To:}\\
% Name\\
% Affiliation\\
%contact info\\
\end{flushleft}
\end{minipage}~
\begin{minipage}{0.4\textwidth}
\begin{flushright} \large
\emph{Submitted By :} \\
Tyler Bendele
\end{flushright}
\end{minipage}\\[2 cm]
% \includegraphics[scale = 0.5]{PICMathLogo.png}
\end{titlepage}
%%%%%%%%%%%%%%%%%%%%%%%%%%%%%%%%%%%%%%%%%%%%%%%%%%%%%%%%%%%%%%%%%%%%%%%%%%
%%%%%%%%%%%%%%%
\tableofcontents
\pagebreak
%%%%%%%%%%%%%%%%%%%%%%%%%%%%%%%%%%%%%%%%%%%%%%%%%%%%%%%%%%%%%%%%%%%%%%%%%%
%%%%%%%%%%%%%%%
\renewcommand{\thesection}{\arabic{section}}
\section{Introduction}
The purpose of this lab is to apply what we have learned through out the
course towards a practical application by designing a filter.

In this project, we are working for an aircraft company and must be able to
filter feedback signals from air crafts in order to enhance 
the performance of the positioning system. The needed information for the
position measurement is located within the range of 1.8kHz to 2.0kHz. We
are tasked with creating a filter that will accurately find the signals
within this range of frequencies while also taking into account the low
frequency vibrations from the building's ventilation system.

\section{Equations}
The Equation shown below is the Transfer function of the filter design I
decided to use, the values of the passive components can be seen in the
circuit design.
\begin{equation}
    H(z) = \frac{\frac{1}{C_{2}R_{2}}s}{s^{2} + (\frac{1}{C_{1}R{1}} + \frac{1}{C_{1}R_{2}} + \frac{1}{C_{2}R_{2}})s + \frac{1}{C_{1}C_{2}R_{1}R_{2}}}
\end{equation}

\section{Filter Design}
\begin{figure}[H]
\begin{center}
\caption{RC Bandpass Filter}
\includegraphics[scale=0.60]{RC Bandpass Filter Design.png}
\end{center}
\end{figure}
 
\section{Methodology}
\subsection{Tasks 1}
To start, I first wanted to get a better picture of the Noisy signal that
was going to be passed through the filter. I plotted the signal using
the given starter code and excel sheet containing the noise.

\begin{lstlisting}[language=Python]
# load input signal
df = pd.read_csv('NoisySignal.csv')

t = df ['0']. values
sensor_sig = df ['1']. values

plt.figure ( figsize = (10 , 7) )
plt.plot (t , sensor_sig )
plt.grid ()
plt.title ('Noisy Input Signal')
plt.xlabel ('Time [s]')
plt.ylabel ('Amplitude [V]')
plt.show ()

# your code starts here , good luck
# Task 1

fs = 1e6
steps = 1e-2
ta = np.arange(1e-5, 2, steps)
\end{lstlisting}

Once I had a good picture of the noise, I wanted to identify the noise
magnitudes and corresponding frequencies. This was done by putting the
noisy signal through the fast fourier transform function. I then plotted the
magnitude and phase over the entire range, before the expected frequencies,
the expected frequency range, and after the expected frequencies. To 
attenuate any frequencies greater than 100kHz, I simply set the range 
of the graph to end at 100kHz. To account for the 0.05V given off by the 
low frequency vibrations of the building I modified the phase to be zero
when the magnitude was less than 0.05V. I then created the graphs of the
magnitude and phase for each of the specified ranges using the single plot
function given for the .stem function. These graphs can be seen plotted in
the results section.

\begin{lstlisting}[language=Python]
freq1, X_mag1, X_phi1 = cleanfft(sensor_sig)

def make_stem ( ax ,x ,y , color ='k', style ='solid', label ='', linewidths =2.5 ,** kwargs ) :
    ax.axhline ( x [0] , x [ -1] ,0 , color ='r')
    ax.vlines (x , 0 ,y , color = color , linestyles = style , label = label , linewidths = linewidths )
    ax.set_ylim ([1.05* y . min () , 1.05* y . max () ])


# Magnitude and Phase of Full Range
fig , ax = plt . subplots ( figsize =(10 , 7) )
make_stem( ax , freq1 , X_mag1)
plt.title('Clean FFt Magnitude of Noisy signal')
plt.ylabel('|x(t)|')
plt.xlabel('f[Hz]')
plt.show()

fig , ax = plt . subplots ( figsize =(10 , 7) )
make_stem( ax , freq1 , X_phi1)
plt.title('Clean FFt Phase of Noisy signal')
plt.ylabel('/_x(t)')
plt.xlabel('f[Hz]')
plt.show()
\end{lstlisting}

\subsection{Task 2}
Once I knew what I was looking for, I was able to start creating
my filter design. To do this, I first did a bit of research. The
simplest design to understand that I found was a bandpass filter
that combines a high pass and a low pass filter. This was done using
two different RC configurations as shown in the circuit I created above.
I then needed to find the Transfer function of the filter through
a bit of circuit analysis and algebra. To check my transfer function I
first plugged in into python to see how the bode plot of the function
looked. I also ran the noise through the filter to see how that looked too.
However, when I did this it looked kind of funky almost like I forgot to
account for half of the filter. I ended up looking up the transfer
function for the specific filter I was using to compare it to the
one I calculated. It turns out they had noticeable differences, so
I ended up plugging this new function into python and it gave a much
better result. The function I came up with and the RC Bandpass filter
circuit I used can be seen in earlier sections. The test code can be seen
down below.
\begin{lstlisting}[language=Python]
# Part 2 Filter Test
R1 = 10e3
R2 = 10e3
C1 = 8.85e-9      
C2 = 7.96e-9          

num1 = [(1/C2*R2), 0]
den1 = [1, ((1/(C1*R1)) + (1/(C1*R2)) + (1/(C2*R2))), (1/(C1*C2*R1*R2))]

steps = 1
w = np.arange(1e3, 1e6 +steps, steps)

sys = scipy.signal.TransferFunction( num1, den1)
w1, mag1, phase1 = scipy.signal.bode(sys, w)

plt.figure(figsize=(10,7))
plt.subplot(1,1,1)
plt.semilogx(w1, mag1)
plt.title('Magnitude Transfer Function using Bode Function')                           
plt.ylabel('|H(jw)|')
plt.xlabel('w')
plt.grid(True)

num2, den2 = scipy.signal.bilinear(num1, den1, fs)
y = scipy.signal.lfilter(num2, den2, sensor_sig)

plt.figure(figsize=(10,7))
plt.subplot(1,1,1)
plt.plot(t, y)
plt.grid(True)
plt.title('Part 2 Noisy Signal Filtered Test')                           
plt.ylabel('y(t)')
plt.xlabel('t')
plt.show()
\end{lstlisting}

\subsection{Task 3}
For the third task, I needed to generate the bode plots of the filter to
make sure it was attenuated to the specifications of the handout.
Unfortunately, I found this part difficult to accomplish because of how
my plots turned out. The first bode plot I create showed the entire range of
the filter and it seemed to give what I was looking for, with the
curve focused on the range of 1.8kHz to 2.0kHz. The third bode plot is
also useful because it better shows the curve that I was looking for over
the wanted range. The difficult part of this task was reading the graphs
in order to attenuate the noise. This needed to be done in order to meet
the requirements of the position measurement being attenuated to less than
-0.03dB, the low-frequency vibration noise being attenuated by at least
-30dB, and the switching amplifer noise to be attenuated by at least -21dB.
This part was difficult because the y component wasn't computed in decibels
for some reason. Either that, or there was a problem in my design, which is
also very possible. Because of this complication I did not adjust the
resistors at all in order to tighten the spread, which could lead to
the attenuation requirements not being met.

\begin{lstlisting}[language=Python]
# Part 3
sys = con.TransferFunction ( num1 , den1 )
plt.figure(figsize=(10,7))
_ = con.bode( sys , w*2*np.pi , dB = True , Hz = True , deg = True , Plot = True )

plt.figure(figsize=(10,7))
_ = con.bode( sys , np.arange(1e-5, 1800 +steps, steps)*2*np.pi , dB = True , Hz = True , deg = True , Plot = True )

plt.figure(figsize=(10,7))
_ = con.bode( sys , np.arange(1800, 2000 +steps, steps)*2*np.pi , dB = True , Hz = True , deg = True , Plot = True )

plt.figure(figsize=(10,7))
_ = con.bode( sys , np.arange(2000, 100000 +steps, steps)*2*np.pi , dB = True , Hz = True , deg = True , Plot = True )
\end{lstlisting}

\subsection{Task 4}
For the final task, I needed to filter the noisy sensor signal using the
filter I created and then show how it has been correctly attenuated at the
appropriate frequencies. To do this, I placed the filtered function from
part 2 into the fast fourier transform from part 1 and then just
graph each of the specified ranges again. Since the magnitude is the
important part that should be looked at using the filter, I decided to
only include the magnitudes of the filtered output in the results section.
Example code for how I plotted each function can be seen down below.
\begin{lstlisting}[language=Python]
# Part 4
freq2, X_mag2, X_phi2 = cleanfft(y)
# Magnitude and Phase of Full Range
fig , ax = plt . subplots ( figsize =(10 , 7) )
make_stem( ax , freq2 , X_mag2)
plt.title('Clean FFt Magnitude of Noisy signal')
plt.ylabel('|x(t)|')
plt.xlabel('f[Hz]')
plt.show()

fig , ax = plt . subplots ( figsize =(10 , 7) )
make_stem( ax , freq2 , X_phi2)
plt.title('Clean FFt Phase of Noisy signal')
plt.ylabel('/_x(t)')
plt.xlabel('f[Hz]')
plt.show()
\end{lstlisting}
\section{Results}
\begin{figure}[H]
\begin{center}
\caption{Noise}
\includegraphics[scale=0.50]{NoisySig.png}
\end{center}
\end{figure}

\subsection{Task 1}
\begin{figure}[H]
\begin{center}
\caption{Full Unfiltered Range Magnitude}
\includegraphics[scale=0.50]{unfilteredFullMag.png}
\end{center}
\end{figure}

\begin{figure}[H]
\begin{center}
\caption{Full Unfiltered Range Phase}
\includegraphics[scale=0.50]{unfilteredFullPhase.png}
\end{center}
\end{figure}

\begin{figure}[H]
\begin{center}
\caption{Unfiltered Range 0 to 1800 Magnitude}
\includegraphics[scale=0.50]{unfilteredBeforeMag.png}
\end{center}
\end{figure}

\begin{figure}[H]
\begin{center}
\caption{Unfiltered Range 0 to 1800 Phase}
\includegraphics[scale=0.50]{unfilteredBeforePhase.png}
\end{center}
\end{figure}

\begin{figure}[H]
\begin{center}
\caption{Unfiltered Range 1800 to 2000 Magnitude}
\includegraphics[scale=0.50]{UnfilteredWantMag.png}
\end{center}
\end{figure}

\begin{figure}[H]
\begin{center}
\caption{Unfiltered Range 1800 to 2000 Phase}
\includegraphics[scale=0.50]{unfilteredWantPhase.png}
\end{center}
\end{figure}

\begin{figure}[H]
\begin{center}
\caption{Unfiltered Range 2000 to 100000 Magnitude}
\includegraphics[scale=0.50]{unfilteredAfterMag.png}
\end{center}
\end{figure}

\begin{figure}[H]
\begin{center}
\caption{Unfiltered Range 2000 to 100000 Magnitude}
\includegraphics[scale=0.50]{unfilteredAfterPhase.png}
\end{center}
\end{figure}

\subsection{Task 2}
\begin{figure}[H]
\begin{center}
\caption{Transfer Function Test}
\includegraphics[scale=0.50]{TransferFunc.png}
\end{center}
\end{figure}

\begin{figure}[H]
\begin{center}
\caption{Filter Test}
\includegraphics[scale=0.50]{FilterTest.png}
\end{center}
\end{figure}

\subsection{Task 3}
\begin{figure}[H]
\begin{center}
\caption{Bode Plot of Full Range}
\includegraphics[scale=0.50]{BodeRange.png}
\end{center}
\end{figure}

\begin{figure}[H]
\begin{center}
\caption{Bode Plot of Range 0 to 1800}
\includegraphics[scale=0.50]{BodeBefore.png}
\end{center}
\end{figure}

\begin{figure}[H]
\begin{center}
\caption{Bode Plot of Range 1800 to 2000}
\includegraphics[scale=0.50]{BodeWant.png}
\end{center}
\end{figure}

\begin{figure}[H]
\begin{center}
\caption{Bode Plot of Range 2000 to 100000}
\includegraphics[scale=0.50]{BodeAfter.png}
\end{center}
\end{figure}

\subsection{Task 4}
\begin{figure}[H]
\begin{center}
\caption{Full Filtered Range Magnitude}
\includegraphics[scale=0.50]{FilteredRangeMag.png}
\end{center}
\end{figure}

\begin{figure}[H]
\begin{center}
\caption{Filtered Range 0 to 1800 Magnitude}
\includegraphics[scale=0.50]{FilteredBeforeMag.png}
\end{center}
\end{figure}

\begin{figure}[H]
\begin{center}
\caption{Filtered Range 1800 to 2000 Magnitude}
\includegraphics[scale=0.50]{FilteredWantMag.png}
\end{center}
\end{figure}

\begin{figure}[H]
\begin{center}
\caption{Filtered Range 2000 to 100000 Magnitude}
\includegraphics[scale=0.50]{FilteredAfterMag.png}
\end{center}
\end{figure}

\section{Error Analysis}
I had a couple of problems with this project. I think the
main problem I had was trying to adjust my plot based off
of the bode plots generated. Since all of the bode plots
except for the wanted range give large values for the
magnitude, it makes me wonder if it didn't actually
plot in decibels. Luckily, the most important bode plot,
the range of 1.8 kHz to 2.0 kHz, gave exactly what I was
expecting. Because of the uncertainty seen in these
plots, I did not adjust my passive filter components at
all. However, based on the plots I was able to get out of the
filtered output, I think it filtered pretty well. It only
has a couple of smaller noises outside of the wanted range
that made it through the filter. The phase, I received
from the filter had quite a bit of unwanted noise in it.
Fortunately, the goal was to be able to read the 
magnitudes of the noise within the expected frequencies.
I might have been able to attenuate the graphs better if
I were to have used an RLC design for my Bandpass filter.
I believe this is true, because The RLC circuit given from
lab 10, seems to make the range of wanted frequencies
tighter.

\section{Questions}
1. Earlier this semester, you were asked what you personally wanted to get out of taking this
course. Do you feel like that personal goal was met? Why or why not?

Yes, I definitely think I was able to get what I wanted
out of the course. I wanted to learn as much as possible
and I definitely learned a lot through this class. In 
fact, it has somewhat sparked my curiosity to learn
more about signals and systems.

2. Please fill out the course feedback survey, I will read every word and very much appreciate
the feedback.

3. Good luck in the rest of your education and career!

Thanks, you too!

\section{Conclusion}
I believe that for the most part this lab went pretty 
well. I was able to apply a lot of what I have learned
over the semester to this project and actually have it
apply to a real world concept. I also think my filter
turned out good, however there are quite a few
improvements that can be made. For one thing, I could
adjust the fft function to better account for the phase
since there was quite a bit of noise in my filtered output
specifically for the phase. I could also adjust the
passive components of the circuit to more tightly
fit the wanted range of frequencies. Besides, this
I think the filter turned out pretty well.


\end{document}
%This template was created by Roza Aceska.