\documentclass{article}
%%%%%%%%%%%%%%%%%%%%%%%%%%%%%%%%%%%%%%%%%%%%%%%%%%%%%%%%%%%%%%%%
%                                                              %
% Tyler Bendele                                                %
% ECE351 and Section 51                                        %
% Lab 1                                                        %
% Due 1/25/2022                                                %
% Introductory lab to Python and Latex                         %
%                                                              %
%%%%%%%%%%%%%%%%%%%%%%%%%%%%%%%%%%%%%%%%%%%%%%%%%%%%%%%%%%%%%%%%
% Language setting
% Replace `english' with e.g. `spanish' to change the document language
\usepackage[english]{babel}

% Set page size and margins
% Replace `letterpaper' with`a4paper' for UK/EU standard size
\usepackage[letterpaper,top=2cm,bottom=2cm,left=3cm,right=3cm,marginparwidth=1.75cm]{geometry}

% Useful packages
\usepackage{amsmath}
\usepackage{graphicx}
\usepackage[colorlinks=true, allcolors=blue]{hyperref}

\title{Lab 1: Introduction to Python 3.x and Latex}
\author{Tyler Bendele}

\begin{document}
\maketitle


\section{Part 1 Summary}
In the first part, the lab has us go through an interactive tutorial and read the Spyder keyboard shortcut cheat sheet. For some reason the interactive tutorial link didn't work. After looking over the shortcut cheat sheet, it has us create a new file in Spyder to test code shown in other sections.

\section{Part 2 Summary}
The sample codes start out pretty basic, showing how to print variables, text, and variables with text. It also shows a little bit of arithmetic and exponents. It then goes on to representing lists and arrays. Next, it shows that imports can be nicknamed in the code, so it is easier and faster to write. After this tip, the lab goes on to explain many more ways that arrays can be represented. The lab moves on to show how to create graphs. Then it goes over how imaginary and mixed numbers show up and errors that occur if not done correctly. This part then ends showing many other packages that will be used this semester as well as other helpful commands.

\section{Part 3 Summary}
This part of the lab starts by showing how indentation should be used only for if and else statements and function definitions. It then states that docstrings should be used for defining certain functions. Another rule of thumb is that lines shouldn't exceed 79 characters, so it is good to wrap lines by using parenthesis, brackets, and braces. The next step states how comments should be formatted. To end part 3, the lab goes over spacing between operators and naming conventions.

\section{Part 4 summary}
This final part of the lab, encourages us to read through the LATEX cheat sheets and find a template that would be useful for lab reports in this class. It also gives us useful code for package imports in this lab.

\section{Questions}
\begin{enumerate}
\item What course are you most excited for in your degree? Which course have you enjoyed the most so far?
\newline I am actually not sure what class I am most excited for in my degree. For this semester, I think I am most excited for ECE350 Signals and Systems. I am also pretty excited for Linear Algebra. The course that I enjoyed the content the most in so far is ECE310 Microelectronics. I wasn't a big fan of the intense workload that came with it though.
\item Leave any feedback on the clarity of expectations, instructions, and deliverables.
\newline As far as I can tell the clarity of the expectations, instructions, and deliverables seem pretty clear. I am sure I will have many questions later though.
\end{enumerate}

\end{document}