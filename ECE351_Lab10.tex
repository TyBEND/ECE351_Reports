%%%%%%%%%%%%%%%%%%%%%%%%%%%%%%%%%%%%%%%%%%%%%%%%%%%%%%%%%%%%%%%%
%                                                              %
% Tyler Bendele                                                %
% ECE351 and Section 51                                        %
% Lab 10                                                       %
% Due Date: April 5, 2022                                      %
% Frequency Response                                           %
%                                                              %
%%%%%%%%%%%%%%%%%%%%%%%%%%%%%%%%%%%%%%%%%%%%%%%%%%%%%%%%%%%%%%%%
%%% DOCUMENT PREAMBLE %%%
\documentclass[12pt]{report}
\usepackage[english]{babel}
%\usepackage{natbib}
\usepackage{url}
\usepackage[utf8x]{inputenc}
\usepackage{amsmath}
\usepackage{graphicx}
\graphicspath{{images/}}
\usepackage{parskip}
\usepackage{fancyhdr}
\usepackage{vmargin}
\usepackage{listings}
\usepackage{hyperref}
\usepackage{xcolor}
\usepackage{float}
\definecolor{codegreen}{rgb}{0,0.6,0}
\definecolor{codegray}{rgb}{0.5,0.5,0.5}
\definecolor{codeblue}{rgb}{0,0,0.95}
\definecolor{backcolour}{rgb}{0.95,0.95,0.92}
\lstdefinestyle{mystyle}{
backgroundcolor=\color{backcolour},
commentstyle=\color{codegreen},
keywordstyle=\color{codeblue},
numberstyle=\tiny\color{codegray},
stringstyle=\color{codegreen},
basicstyle=\ttfamily\footnotesize,
breakatwhitespace=false,
breaklines=true,
captionpos=b,
keepspaces=true,
numbers=left,
numbersep=5pt,
showspaces=false,
showstringspaces=false,
showtabs=false,
tabsize=2
}
\lstset{style=mystyle}
\setmarginsrb{3 cm}{2.5 cm}{3 cm}{2.5 cm}{1 cm}{1.5 cm}{1 cm}{1.5 cm}
\title{Lab 10: Frequency Response}
% Title
\author{Tyler Bendele}
% Author
\date{April 5, 2022}
% Date
\makeatletter
\let\thetitle\@title
\let\theauthor\@author
\let\thedate\@date
\makeatother
\pagestyle{fancy}
\fancyhf{}
\rhead{\theauthor}
\lhead{\thetitle}
\cfoot{\thepage}
%%%%%%%%%%%%%%%%%%%%%%%%%%%%%%%%%%%%%%%%%%%%
\begin{document}
%%%%%%%%%%%%%%%%%%%%%%%%%%%%%%%%%%%%%%%%%%%%%%%%%%%%%%%%%%%%%%%%%%%%%%%%%%
%%%%%%%%%%%%%%%
\begin{titlepage}
\centering
\vspace*{0.5 cm}
% \includegraphics[scale = 0.075]{bsulogo.png}\\[1.0 cm] % 

\begin{center}    \textsc{\Large   ECE 351 - Section \#51 }\\[2.0 cm]
\end{center}% University Name
\textsc{\Large Signals and Systems  }\\[0.5 cm] % Course 

\rule{\linewidth}{0.2 mm} \\[0.4 cm]
{ \huge \bfseries \thetitle}\\
\rule{\linewidth}{0.2 mm} \\[1.5 cm]
\begin{minipage}{0.4\textwidth}
\begin{flushleft} \large
% \emph{Submitted To:}\\
% Name\\
% Affiliation\\
%contact info\\
\end{flushleft}
\end{minipage}~
\begin{minipage}{0.4\textwidth}
\begin{flushright} \large
\emph{Submitted By :} \\
Tyler Bendele
\end{flushright}
\end{minipage}\\[2 cm]
% \includegraphics[scale = 0.5]{PICMathLogo.png}
\end{titlepage}
%%%%%%%%%%%%%%%%%%%%%%%%%%%%%%%%%%%%%%%%%%%%%%%%%%%%%%%%%%%%%%%%%%%%%%%%%%
%%%%%%%%%%%%%%%
\tableofcontents
\pagebreak
%%%%%%%%%%%%%%%%%%%%%%%%%%%%%%%%%%%%%%%%%%%%%%%%%%%%%%%%%%%%%%%%%%%%%%%%%%
%%%%%%%%%%%%%%%
\renewcommand{\thesection}{\arabic{section}}
\section{Introduction}
The purpose of this lab is to become familiar with using the frequency
response tools and Bode plots in Python.
\section{Equations}
The first equation shown down below is the the transfer function that
the RLC circuit gave. The next two equations are the magnitude and 
phase functions found from the Transfer function during the prelab
and used for the first part of the lab. The fourth equation is the
signal that was used for the second part of the lab.
\begin{equation}
    H(s) = \frac{\frac{1}{RC}s}{s^{2} + \frac{1}{RC}s + \frac{1}{LC}}
\end{equation}
\begin{equation}
    |H(j\omega)| = \frac{\frac{1}{RC}\omega}{\sqrt{(-\omega^{2} + \frac{1}{LC})^{2}+(\frac{1}{RC}\omega)^2}}
\end{equation}
\begin{equation}
    \angle H(j\omega) = 90^{\circ} - tan^{-1}(\frac{\frac{1}{RC}w}{(-\omega + \frac{1}{LC})})
\end{equation}
\begin{equation}
     x(t) = cos(2\pi \cdot 100t) + cos(2\pi \cdot 3024t) + sin(2\pi \cdot 50000t)
\end{equation}
\section{Methodology}
\subsection{Part 1}
For the first task I created user defined functions of the magnitude and
phase expressions that I found in the prelab. Since I needed to plot on the
logarithmic scale I made sure to convert the magnitude to decibels and
the phase to degrees. I then plotted the magnitude and phase on different
graphs from \begin{math} 10^{3} rad/s \le \omega \le 10^{6} rad/s 
\end{math} using the matplotlib.pyplot.semilogx() function to plot. 
The code for this can be seen down below.
\begin{lstlisting}[language=Python]
# Part 1 Task 1
steps = 1
w = np.arange(1e3, 1e6 +steps, steps)

def hm(w, R, L, C):
    m = (w/(R*C)) / np.sqrt((w**4) + ((((1/(R*C))**2) - (2/L*C))*(w**2)) + ((1/(L*C))**2))
    m = 20*np.log10(m)
    return m


def hp(w, R, L, C):
    a = np.pi/2 - np.arctan(((w/(R*C)))/((1/(L*C)) - (w**2)))
    a = np.degrees(a)
    for i in range(len(w)):
        if a[i] > 90:
            a[i] = a[i]-180
    return a

R = 1000
L = 27e-3
C = 100e-9

h1m = hm(w, R, L, C)
h1p = hp(w, R, L, C)

plt.figure(figsize=(10,7))
plt.subplot(2,1,1)
plt.semilogx(w, h1m)
plt.title('Magnitude and Phase of Transfer Function')                           
plt.ylabel('|H(jw)|')
plt.grid(True)

plt.subplot(2,1,2)
plt.semilogx(w, h1p)
plt.grid(True)
plt.xlabel('w') 
plt.ylabel('/_H(jw)') 
plt.show()
\end{lstlisting}
The second task for part one just had us plot the functions again
except by using the scipy.signal.bode function. Then the third task
had us plot the function in Hz instead of rad/s, which was done using
the control package Transfer Function.
\begin{lstlisting}[language=Python]
# P1 Task 2
sys = scipy.signal.TransferFunction( [1/(R*C), 0], [1, 1/(R*C), 1/(L*C)])
w1, mag1, phase1 = scipy.signal.bode(sys, w)

plt.figure(figsize=(10,7))
plt.subplot(2,1,1)
plt.semilogx(w1, mag1)
plt.title('Magnitude and Phase of Transfer Function using Bode Function')                           
plt.ylabel('|H(jw)|')
plt.grid(True)

plt.subplot(2,1,2)
plt.semilogx(w1, phase1)
plt.grid(True)
plt.xlabel('w') 
plt.ylabel('/_H(jw)') 
plt.show()

# P1 Task 3
num1 = [1/(R*C), 0]
den1 = [1, 1/(R*C), 1/(L*C)]

sys = con.TransferFunction ( num1 , den1 )
_ = con.bode( sys , w , dB = True , Hz = True , deg = True , Plot = True )
# use _ = ... to suppress the output
\end{lstlisting}

\subsection{Part 2}
For the second part of the lab, we are given a signal to plot
which is the expression shown in equation 4. This signal is plotted with
a range of \begin{math} 0 \le t \le 0.01s \end{math} and a step size of
1/fs. Finally, for the second, third, and fourth tasks of part 2,
I am tasked with passing this signal through the RLC filter from the
first part. This was done by first converting the transfer function into
the z domain using the scipy.signal.bilinear() function. I was then able
to pass the signal through the filter using the scipy.signal.lfilter()
function. Afterwards, I plotted the function.
\begin{lstlisting}[language=Python]
# Part 2 task 1
fs = 50000
steps = 1/fs
t = np.arange(0, 0.01 + steps, steps)

x = np.cos(2*np.pi*100*t) + np.cos(2*np.pi*3024*t) + np.sin(2*np.pi*50000*t)

plt.figure(figsize=(10,7))
plt.subplot(1,1,1)
plt.plot(t, x)
plt.grid(True)
plt.title('Three Frequency Plot')                           
plt.ylabel('x(t)')
plt.xlabel('t')
plt.show()

# P2 Tasks 2, 3, and 4

num2, den2 = scipy.signal.bilinear(num1, den1, fs)
y = scipy.signal.lfilter(num2, den2, x)

plt.figure(figsize=(10,7))
plt.subplot(1,1,1)
plt.plot(t, y)
plt.grid(True)
plt.title('Three Frequency Plot with lfilter')                           
plt.ylabel('x(t)')
plt.xlabel('t')
plt.show()
\end{lstlisting}
\section{Results}
\subsection{Part 1}
\begin{figure}[H]
\begin{center}
\caption{Task 1}
\includegraphics[scale=0.60]{pt1task1.png}
\end{center}
\end{figure}

\begin{figure}[H]
\begin{center}
\caption{Task 2}
\includegraphics[scale=0.60]{pt1task2.png}
\end{center}
\end{figure}

\begin{figure}[H]
\begin{center}
\caption{Task 3}
\includegraphics[scale=0.90]{pt1task3.png}
\end{center}
\end{figure}

\subsection{Part 2}

\begin{figure}[H]
\begin{center}
\caption{Task}
\includegraphics[scale=0.55]{pt2task1.png}
\end{center}
\end{figure}

\begin{figure}[H]
\begin{center}
\caption{Tasks 2, 3, and 4}
\includegraphics[scale=0.55]{pt2task4.png}
\end{center}
\end{figure}

\section{Error Analysis}
I don't think I had too many errors happen with this lab. It took me
a while to figure out how to make the magnitude and phase functions
work. After that everything seemed to go pretty smoothly.

\section{Questions}
1. Explain how the filter and filtered output in Part 2 makes sense given
the Bode plots from Part 1. Discuss how the filter modifies specific
frequency bands, in Hz.

I am not sure if I am reading this correctly but the graph of the filter
in Task 3 looks like it filters out the frequency at 3024 rad/s. This
makes sense when looking at the filtered output in Part 2, since the
two functions that are left afterwards are a sin and a cos function
which cancel each other out.


2. Discuss the purpose and workings of scipy.signal.bilinear() and
scipy.signal.lfilter()

The purpose of the bilinear() function is to return a digital
IIR filter from an analog filter using a bilinear transform.
The lfilter() is used to filter data using a IIR or FIR filter.

3. What happens if you use a different sampling frequency in
scipy.signal.bilinear() than you used for the time-domain signal?

It seems to change the amplitude of signal that gets filtered.

4. Leave any feedback on the clarity of lab tasks, expectations, and
deliverables.

Everything for this lab seemed pretty clear. I think the only thing
that was kind of confusing is that there were three tasks in part 2
that led up to one result.

\section{Conclusion}
I think this lab went pretty well. It introduced many useful tools
that can be used when filtering a signal. It gave us practice in creating
magnitude and phase functions based off of a Transform function. It 
introduced how to create a bode plot, and it also introduced how to run a 
signal through a filter. 

\end{document}
%This template was created by Roza Aceska.