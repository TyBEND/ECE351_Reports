%%%%%%%%%%%%%%%%%%%%%%%%%%%%%%%%%%%%%%%%%%%%%%%%%%%%%%%%%%%%%%%%
%                                                              %
% Tyler Bendele                                                %
% ECE351 and Section 51                                        %
% Lab 11                                                       %
% Due Date: April 12, 2022                                     %
% Z-Transform Operation                                        %
%                                                              %
%%%%%%%%%%%%%%%%%%%%%%%%%%%%%%%%%%%%%%%%%%%%%%%%%%%%%%%%%%%%%%%%
%%% DOCUMENT PREAMBLE %%%
\documentclass[12pt]{report}
\usepackage[english]{babel}
%\usepackage{natbib}
\usepackage{url}
\usepackage[utf8x]{inputenc}
\usepackage{amsmath}
\usepackage{graphicx}
\graphicspath{{images/}}
\usepackage{parskip}
\usepackage{fancyhdr}
\usepackage{vmargin}
\usepackage{listings}
\usepackage{hyperref}
\usepackage{xcolor}
\usepackage{float}
\definecolor{codegreen}{rgb}{0,0.6,0}
\definecolor{codegray}{rgb}{0.5,0.5,0.5}
\definecolor{codeblue}{rgb}{0,0,0.95}
\definecolor{backcolour}{rgb}{0.95,0.95,0.92}
\lstdefinestyle{mystyle}{
backgroundcolor=\color{backcolour},
commentstyle=\color{codegreen},
keywordstyle=\color{codeblue},
numberstyle=\tiny\color{codegray},
stringstyle=\color{codegreen},
basicstyle=\ttfamily\footnotesize,
breakatwhitespace=false,
breaklines=true,
captionpos=b,
keepspaces=true,
numbers=left,
numbersep=5pt,
showspaces=false,
showstringspaces=false,
showtabs=false,
tabsize=2
}
\lstset{style=mystyle}
\setmarginsrb{3 cm}{2.5 cm}{3 cm}{2.5 cm}{1 cm}{1.5 cm}{1 cm}{1.5 cm}
\title{Lab 11: Z-Transform Operation}
% Title
\author{Tyler Bendele}
% Author
\date{April 12, 2022}
% Date
\makeatletter
\let\thetitle\@title
\let\theauthor\@author
\let\thedate\@date
\makeatother
\pagestyle{fancy}
\fancyhf{}
\rhead{\theauthor}
\lhead{\thetitle}
\cfoot{\thepage}
%%%%%%%%%%%%%%%%%%%%%%%%%%%%%%%%%%%%%%%%%%%%
\begin{document}
%%%%%%%%%%%%%%%%%%%%%%%%%%%%%%%%%%%%%%%%%%%%%%%%%%%%%%%%%%%%%%%%%%%%%%%%%%
%%%%%%%%%%%%%%%
\begin{titlepage}
\centering
\vspace*{0.5 cm}
% \includegraphics[scale = 0.075]{bsulogo.png}\\[1.0 cm] % 

\begin{center}    \textsc{\Large   ECE 351 - Section \#51 }\\[2.0 cm]
\end{center}% University Name
\textsc{\Large Signals and Systems  }\\[0.5 cm] % Course 

\rule{\linewidth}{0.2 mm} \\[0.4 cm]
{ \huge \bfseries \thetitle}\\
\rule{\linewidth}{0.2 mm} \\[1.5 cm]
\begin{minipage}{0.4\textwidth}
\begin{flushleft} \large
% \emph{Submitted To:}\\
% Name\\
% Affiliation\\
%contact info\\
\end{flushleft}
\end{minipage}~
\begin{minipage}{0.4\textwidth}
\begin{flushright} \large
\emph{Submitted By :} \\
Tyler Bendele
\end{flushright}
\end{minipage}\\[2 cm]
% \includegraphics[scale = 0.5]{PICMathLogo.png}
\end{titlepage}
%%%%%%%%%%%%%%%%%%%%%%%%%%%%%%%%%%%%%%%%%%%%%%%%%%%%%%%%%%%%%%%%%%%%%%%%%%
%%%%%%%%%%%%%%%
\tableofcontents
\pagebreak
%%%%%%%%%%%%%%%%%%%%%%%%%%%%%%%%%%%%%%%%%%%%%%%%%%%%%%%%%%%%%%%%%%%%%%%%%%
%%%%%%%%%%%%%%%
\renewcommand{\thesection}{\arabic{section}}
\section{Introduction}
The purpose of this lab is to analyze a discrete system using Z transforms
by hand and by using Python's built-in functions developed by Christopher 
Felton.
\section{Equations}
The first equation shown is the causal function that was considered.
The second equation is the Z-transform of the causal function that was
done by hand. The third equation is the partial fraction expansion result
that was also found by hand.
\begin{equation}
    y[k] = 2x[k] - 40x[k-1] + 10y[k-1] - 16y[k-2]
\end{equation}
\begin{equation}
    H(z) = \frac{2 (1 - 20z^{-1})}{(1 - 8z^{-1})(1 - 2z^{-1})}
\end{equation}
\begin{equation}
    h[k] = -4(8^{k})u[k] + 6(2^{k})u[k]
\end{equation}
\section{Methodology}
\subsection{Tasks 1 \& 2}
For the first task I was given the casual equation shown in equation 1.
I was then tasked with finding the z transform. This was done by finding
the z-transform of all of the individual parts and then putting the output
over the input. This Z-transform function is shown in equation 2.

For the second task, I needed to use partial fraction expansion on and then
use the inverse z transform which gives the third equation.
\subsection{Task 3}
For the third task I used the scipy.signal.residuez() function to confirm
that my partial fraction expansion was correct, this was done using the
code shown down below.
\begin{lstlisting}[language=Python]
# Task 3

num = [2, -40] # Creates a matrix for the numerator
den = [1, -10, 16] # Creates a matrix for the denominator

R, P, K = sig.residuez(num, den)

print("Partial Fraction Expansion:")
print("Residues:", R)
print("Poles:", P)
print("Coefficents", K)
\end{lstlisting}

\subsection{Task 4}
For the fourth task I used the zplane function that was created by
Christopher Felton.This function was used to obtain the pole-zero plot
of the H(z) function.
\begin{lstlisting}[language=Python]
# Task 4

#%% Zplane function
#
# Copyright (c) 2011 Christopher Felton
#
# This program is free software: you can redistribute it and/or modify
# it under the terms of the GNU Lesser General Public License as published by
# the Free Software Foundation, either version 3 of the License, or
# (at your option) any later version.
#
# This program is distributed in the hope that it will be useful,
# but WITHOUT ANY WARRANTY; without even the implied warranty of
# MERCHANTABILITY or FITNESS FOR A PARTICULAR PURPOSE.  See the
# GNU Lesser General Public License for more details.
#
# You should have received a copy of the GNU Lesser General Public License
# along with this program.  If not, see <http://www.gnu.org/licenses/>.
#
# The following is derived from the slides presented by
# Alexander Kain for CS506/606 "Special Topics: Speech Signal Processing"
# CSLU / OHSU, Spring Term 2011.
#
#
#
# Modified by Drew Owens in Fall 2018 for use in the University of Idaho's 
# Department of Electrical and Computer Engineering Signals and Systems I Lab
# (ECE 351)
#
# Modified by Morteza Soltani in Spring 2019 for use in the ECE 351 of the U of
# I.
#
# Modified by Phillip Hagen in Fall 2019 for use in the University of Idaho's  
# Department of Electrical and Computer Engineering Signals and Systems I Lab 
# (ECE 351)
    
def zplane(b,a,filename=None):
    """Plot the complex z-plane given a transfer function.
    """
    import numpy as np
    import matplotlib.pyplot as plt
    from matplotlib import patches    
    
    # get a figure/plot
    ax = plt.subplot(1,1,1)
    # create the unit circle
    uc = patches.Circle((0,0), radius=1, fill=False,
                        color='black', ls='dashed')
    ax.add_patch(uc)
    # The coefficients are less than 1, normalize the coeficients
    if np.max(b) > 1:
        kn = np.max(b)
        b = np.array(b)/float(kn)
    else:
        kn = 1
    if np.max(a) > 1:
        kd = np.max(a)
        a = np.array(a)/float(kd)
    else:
        kd = 1
        
    # Get the poles and zeros
    p = np.roots(a)
    z = np.roots(b)
    k = kn/float(kd)
    
    # Plot the zeros and set marker properties    
    t1 = plt.plot(z.real, z.imag, 'o', ms=10,label='Zeros')
    plt.setp( t1, markersize=10.0, markeredgewidth=1.0)
    # Plot the poles and set marker properties
    t2 = plt.plot(p.real, p.imag, 'x', ms=10,label='Poles')
    plt.setp( t2, markersize=12.0, markeredgewidth=3.0)
    ax.spines['left'].set_position('center')
    ax.spines['bottom'].set_position('center')
    ax.spines['right'].set_visible(False)
    ax.spines['top'].set_visible(False)
    
    plt.legend()
    # set the ticks
    # r = 1.5; plt.axis('scaled'); plt.axis([-r, r, -r, r])
    # ticks = [-1, -.5, .5, 1]; plt.xticks(ticks); plt.yticks(ticks)
    if filename is None:
        plt.show()
    else:
        plt.savefig(filename)
    
    return z, p, k

zplane(num,den, None)
\end{lstlisting}
\subsection{Task 5}
For the final task I used the scipy.signal.freqz() function to find the
amplitude and frequency of the function. I then plotted the functions as usual.
\begin{lstlisting}[language=Python]
# Task 5
w, h = sig.freqz(num, den, whole = True)

plt.figure(figsize=(10,7))
plt.subplot(2,1,1)
plt.plot(w/np.pi, np.log10(abs(h)))
plt.grid(True)
plt.title('Frequency Response')                           
plt.ylabel('|h(t)| (db)')

plt.subplot(2,1,2)
plt.plot(w/np.pi, np.angle(h))
plt.grid(True)
plt.ylabel('/_H(t) (rad)')
plt.xlabel('Frequency (rad/sample)')
plt.show()
\end{lstlisting}
\section{Results}
\subsection{Part 1}
\begin{figure}[H]
\begin{center}
\caption{Task 3}
\includegraphics[scale=0.90]{task3.png}
\end{center}
\end{figure}

\begin{figure}[H]
\begin{center}
\caption{Task 4}
\includegraphics[scale=0.60]{task4.png}
\end{center}
\end{figure}

\begin{figure}[H]
\begin{center}
\caption{Task 5}
\includegraphics[scale=0.60]{task5.png}
\end{center}
\end{figure}

\section{Error Analysis}
I don't believe I had any errors during this lab.

\section{Questions}
1. Looking at the plot generated in Task 4, is H(z) stable? Explain
why or why not.

No, the plot is not stable because the poles are on the right side 
of the pane and they need to be on the left side of the y-axis in
order to be considered stable.

4. Leave any feedback on the clarity of lab tasks, expectations, and
deliverables.

Everything for this lab seemed pretty clear. I was a little
confused on how to do the z-transforms, but that was
because we hadn't really gone over them in class yet.

\section{Conclusion}
This lab went pretty well. I was able to learn quite
a bit about how to perform z-transforms by hand. I also learned
about how the zplane function can show the stability of a function.


\end{document}
%This template was created by Roza Aceska.