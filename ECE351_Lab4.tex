%%%%%%%%%%%%%%%%%%%%%%%%%%%%%%%%%%%%%%%%%%%%%%%%%%%%%%%%%%%%%%%%
%                                                              %
% Tyler Bendele                                                %
% ECE351 and Section 51                                        %
% Lab 4                                                        %
% Due Date: February 15, 2022                                  %
% Practice finding the step response                           %
%                                                              %
%%%%%%%%%%%%%%%%%%%%%%%%%%%%%%%%%%%%%%%%%%%%%%%%%%%%%%%%%%%%%%%%
%%% DOCUMENT PREAMBLE %%%
\documentclass[12pt]{report}
\usepackage[english]{babel}
%\usepackage{natbib}
\usepackage{url}
\usepackage[utf8x]{inputenc}
\usepackage{amsmath}
\usepackage{graphicx}
\graphicspath{{images/}}
\usepackage{parskip}
\usepackage{fancyhdr}
\usepackage{vmargin}
\usepackage{listings}
\usepackage{hyperref}
\usepackage{xcolor}
\usepackage{float}
\definecolor{codegreen}{rgb}{0,0.6,0}
\definecolor{codegray}{rgb}{0.5,0.5,0.5}
\definecolor{codeblue}{rgb}{0,0,0.95}
\definecolor{backcolour}{rgb}{0.95,0.95,0.92}
\lstdefinestyle{mystyle}{
backgroundcolor=\color{backcolour},
commentstyle=\color{codegreen},
keywordstyle=\color{codeblue},
numberstyle=\tiny\color{codegray},
stringstyle=\color{codegreen},
basicstyle=\ttfamily\footnotesize,
breakatwhitespace=false,
breaklines=true,
captionpos=b,
keepspaces=true,
numbers=left,
numbersep=5pt,
showspaces=false,
showstringspaces=false,
showtabs=false,
tabsize=2
}
\lstset{style=mystyle}
\setmarginsrb{3 cm}{2.5 cm}{3 cm}{2.5 cm}{1 cm}{1.5 cm}{1 cm}{1.5 cm}
\title{Lab 4: System Step Response Using Convolution}
% Title
\author{Tyler Bendele}
% Author
\date{February 15, 2022}
% Date
\makeatletter
\let\thetitle\@title
\let\theauthor\@author
\let\thedate\@date
\makeatother
\pagestyle{fancy}
\fancyhf{}
\rhead{\theauthor}
\lhead{\thetitle}
\cfoot{\thepage}
%%%%%%%%%%%%%%%%%%%%%%%%%%%%%%%%%%%%%%%%%%%%
\begin{document}
%%%%%%%%%%%%%%%%%%%%%%%%%%%%%%%%%%%%%%%%%%%%%%%%%%%%%%%%%%%%%%%%%%%%%%%%%%
%%%%%%%%%%%%%%%
\begin{titlepage}
\centering
\vspace*{0.5 cm}
% \includegraphics[scale = 0.075]{bsulogo.png}\\[1.0 cm] % 

\begin{center}    \textsc{\Large   ECE 351 - Section \#51 }\\[2.0 cm]
\end{center}% University Name
\textsc{\Large Signals and Systems  }\\[0.5 cm] % Course 

\rule{\linewidth}{0.2 mm} \\[0.4 cm]
{ \huge \bfseries \thetitle}\\
\rule{\linewidth}{0.2 mm} \\[1.5 cm]
\begin{minipage}{0.4\textwidth}
\begin{flushleft} \large
% \emph{Submitted To:}\\
% Name\\
% Affiliation\\
%contact info\\
\end{flushleft}
\end{minipage}~
\begin{minipage}{0.4\textwidth}
\begin{flushright} \large
\emph{Submitted By :} \\
Tyler Bendele
\end{flushright}
\end{minipage}\\[2 cm]
% \includegraphics[scale = 0.5]{PICMathLogo.png}
\end{titlepage}
%%%%%%%%%%%%%%%%%%%%%%%%%%%%%%%%%%%%%%%%%%%%%%%%%%%%%%%%%%%%%%%%%%%%%%%%%%
%%%%%%%%%%%%%%%
\tableofcontents
\pagebreak
%%%%%%%%%%%%%%%%%%%%%%%%%%%%%%%%%%%%%%%%%%%%%%%%%%%%%%%%%%%%%%%%%%%%%%%%%%
%%%%%%%%%%%%%%%
\renewcommand{\thesection}{\arabic{section}}
\section{Introduction}
In this lab the goal is to be able to find the step response of
a function using convolution. This will be done both through 
the function used last lab as well as through calculations to 
check the resemblance.
\section{Equations}
The equations shown below includes the 3 functions made in Part 1 and then
used throughout the rest of the lab. The last 3 functions shown are the solutions to the convolution of each function with the
step function.
\begin{equation}
    h_{1}(t) = e^{-2t}(u(t) - u(t-3))
\end{equation}
\begin{equation}
    h_{2}(t) = u(t-2) - u(t-6)
\end{equation}
\begin{equation}
    h_{3}(t) = cos(wt)u(t), w = 2\pi *0.25
\end{equation}
\begin{equation}
    h_{1}(t)\ast u(t) = 0.5\cdot(1 - e^{-2t})u(t) - 0.5\cdot(1 - e^{-2(t-3)})u(t-3) 
\end{equation}
\begin{equation}
    h_{2}(t)\ast u(t) = (t-2)u(t-2) - (t-6)u(t-6)
\end{equation}
\begin{equation}
    h_{3}(t)\ast u(t) = \frac{1}{w}\cdot sin(wt)u(t)
\end{equation}
\section{Methodology}
\subsection{Part 1}
The first part of the lab was pretty simple, all I had to do was create equations 1, 2, and 3 using user defined functions. 
This also required the the step response defined function from
previous labs. I then graphed them all on a single figure with 3 subplots.

\begin{lstlisting}[language=Python]
# First Impulse Response
def h1(t):
    a = np.exp(-2 * t)*(u(t) - u(t - 3))
    return a

# Second Impulse Response
def h2(t):
    b = u(t-2) - u(t-6)
    return b

# Third Impulse Response
def h3(t):
    w = 0.25 * 2 * np.pi
    c = np.cos(w * t)*u(t)
    return c

plt.figure(figsize=(12,8))
plt.subplot(3,1,1)
plt.plot(t, h1(t))
plt.title('Part 1: User Defined Functions') 
                                 
plt.ylabel('h1(t)')
plt.grid(True)

plt.subplot(3,1,2)
plt.plot(t, h2(t))
plt.ylabel('h2(t)') 
plt.grid(which='both')

plt.subplot(3,1,3)
plt.plot(t, h3(t))
plt.grid(True)
plt.xlabel('t') 
plt.ylabel('h3(t)') 
plt.show() 
\end{lstlisting}

\subsection{Part 2}
For Part 2 task 1, we found the step response of
the different functions by convolving them with the unit
step function. This was done, simply by placing them
in the convolution function created in lab 3.
These were then graphed and placed on a single figure
with 3 plots again.
\begin{lstlisting}[language=Python]
y = u(t)
a = h1(t)
b = h2(t)
c = h3(t)

ch1 = my_conv(y, a)*steps
ch2 = my_conv(y, b)*steps
ch3 = my_conv(y, c)*steps

t = np.arange(-10, 10 + steps, steps)
N = len(t)
tExt = np.arange(2 * t[0], 2 * t[N-1] + steps, steps)


plt.figure(figsize = (10, 7))
plt.subplot(3, 1, 1)
plt.plot(tExt, ch1)
plt.grid()
plt.ylabel('h1(t)*u(t)')
plt.title('Unit Step Responses')

plt.subplot(3, 1, 2)
plt.plot(tExt, ch2)
plt.grid()
plt.ylabel('h2(t)*u(t)')

plt.subplot(3, 1, 3)
plt.plot(tExt, ch3)
plt.grid()
plt.ylabel('h3(t)*u(t)') 
plt.xlabel('t')
plt.show()
\end{lstlisting}
For the second task, I hand calculated the convolution between
each function and the step response. Finally, in task three
I used the equations/solutions from task 2 to graph them and
compare to the graphs from the program.
\begin{lstlisting}[language=Python]
t = np.arange(-20, 20 + steps, steps)

def h1c(t):
    x = 0.5*(1 - np.exp(-2*t))*u(t) - 0.5*(1 - np.exp(-2*(t-3)))*u(t-3)
    return x

def h2c(t):
    y = (t-2)*u(t-2) - (t-6)*u(t-6)
    return y
def h3c(t):
    w = 0.25 * 2 * np.pi
    z = (1/w)*np.sin(w*t)*u(t)
    return z

plt.figure(figsize = (10, 7))
plt.subplot(3, 1, 1)
plt.plot(t, h1c(t))
plt.grid()
plt.ylabel('h1(t)*u(t)')
plt.title('Unit Step Response Calculations')

plt.subplot(3, 1, 2)
plt.plot(t, h2c(t))
plt.grid()
plt.ylabel('h2(t)*u(t)')

plt.subplot(3, 1, 3)
plt.plot(t, h3c(t))
plt.grid()
plt.ylabel('h3(t)*u(t)') 
plt.xlabel('t')
plt.show()
\end{lstlisting}

\section{Results}
For the most part, the results are what I expected. Although
there are things that I didn't expect. Specifically, the
results from the calculated equations do not match the
results from the convolution program. I believe that
the convolution function graphed correctly. However,
I also believe my calculated equations, equations 4, 5, and
6, are also correct. This leads me to believe that I
somehow graphed them incorrectly when creating the user
defined functions. This problem only occurred for 
functions 1 and 2 though (equations 4 and 5).
\subsection{Part 1}
\begin{figure}[H]
\begin{center}
\caption{Part 1, Task 2}
\includegraphics[scale=0.55]{Part1Pics.png}
\end{center}
\end{figure}

\subsection{Part 2} {
\begin{figure}[H]
\begin{center}
\caption{Part 2, Task 1, Unit Step Responses}
\includegraphics[scale=0.55]{Part2Task1.png}
\end{center}
\end{figure}

\begin{figure}[H]
\begin{center}
\caption{Part 2, Task 3, Unit Step Response Calculations}
\includegraphics[scale=0.55]{Part2Task3.png}
\end{center}
\end{figure}

}

\section{Error Analysis}
For the most part, my graphs printed correctly. The only
problem is that the ending of the graph for the first two
calculated equations don't quite match the graphs created
from the convolution function from lab 3. I was not able
to find a problem with my calculations, so I believe
the problem is that I some how created the functions
incorrectly in Python. 

\section{Questions}
1. Leave any feedback on the clarity of lab tasks,
expectations, and deliverables.

I think this lab was pretty clear on the instructions and
tasks.
\section{Conclusion}
In this lab, I learned more about how the step response of a 
function is found and how that looks visually. Despite the
graphs for part 1 and 2 not quite matching, I do believe that 
the lab was still a success. I think the lab went pretty well,
and there aren't any recommendations that I can think of 
that may benefit future labs.
\end{document}
%This template was created by Roza Aceska.