%%%%%%%%%%%%%%%%%%%%%%%%%%%%%%%%%%%%%%%%%%%%%%%%%%%%%%%%%%%%%%%%
%                                                              %
% Tyler Bendele                                                %
% ECE351 and Section 51                                        %
% Lab 9                                                        %
% Due Date: March 29, 2022                                     %
% Fast Fourier Transform                                       %
%                                                              %
%%%%%%%%%%%%%%%%%%%%%%%%%%%%%%%%%%%%%%%%%%%%%%%%%%%%%%%%%%%%%%%%
%%% DOCUMENT PREAMBLE %%%
\documentclass[12pt]{report}
\usepackage[english]{babel}
%\usepackage{natbib}
\usepackage{url}
\usepackage[utf8x]{inputenc}
\usepackage{amsmath}
\usepackage{graphicx}
\graphicspath{{images/}}
\usepackage{parskip}
\usepackage{fancyhdr}
\usepackage{vmargin}
\usepackage{listings}
\usepackage{hyperref}
\usepackage{xcolor}
\usepackage{float}
\definecolor{codegreen}{rgb}{0,0.6,0}
\definecolor{codegray}{rgb}{0.5,0.5,0.5}
\definecolor{codeblue}{rgb}{0,0,0.95}
\definecolor{backcolour}{rgb}{0.95,0.95,0.92}
\lstdefinestyle{mystyle}{
backgroundcolor=\color{backcolour},
commentstyle=\color{codegreen},
keywordstyle=\color{codeblue},
numberstyle=\tiny\color{codegray},
stringstyle=\color{codegreen},
basicstyle=\ttfamily\footnotesize,
breakatwhitespace=false,
breaklines=true,
captionpos=b,
keepspaces=true,
numbers=left,
numbersep=5pt,
showspaces=false,
showstringspaces=false,
showtabs=false,
tabsize=2
}
\lstset{style=mystyle}
\setmarginsrb{3 cm}{2.5 cm}{3 cm}{2.5 cm}{1 cm}{1.5 cm}{1 cm}{1.5 cm}
\title{Lab 9: Fast Fourier Transform}
% Title
\author{Tyler Bendele}
% Author
\date{March 29, 2022}
% Date
\makeatletter
\let\thetitle\@title
\let\theauthor\@author
\let\thedate\@date
\makeatother
\pagestyle{fancy}
\fancyhf{}
\rhead{\theauthor}
\lhead{\thetitle}
\cfoot{\thepage}
%%%%%%%%%%%%%%%%%%%%%%%%%%%%%%%%%%%%%%%%%%%%
\begin{document}
%%%%%%%%%%%%%%%%%%%%%%%%%%%%%%%%%%%%%%%%%%%%%%%%%%%%%%%%%%%%%%%%%%%%%%%%%%
%%%%%%%%%%%%%%%
\begin{titlepage}
\centering
\vspace*{0.5 cm}
% \includegraphics[scale = 0.075]{bsulogo.png}\\[1.0 cm] % 

\begin{center}    \textsc{\Large   ECE 351 - Section \#51 }\\[2.0 cm]
\end{center}% University Name
\textsc{\Large Signals and Systems  }\\[0.5 cm] % Course 

\rule{\linewidth}{0.2 mm} \\[0.4 cm]
{ \huge \bfseries \thetitle}\\
\rule{\linewidth}{0.2 mm} \\[1.5 cm]
\begin{minipage}{0.4\textwidth}
\begin{flushleft} \large
% \emph{Submitted To:}\\
% Name\\
% Affiliation\\
%contact info\\
\end{flushleft}
\end{minipage}~
\begin{minipage}{0.4\textwidth}
\begin{flushright} \large
\emph{Submitted By :} \\
Tyler Bendele
\end{flushright}
\end{minipage}\\[2 cm]
% \includegraphics[scale = 0.5]{PICMathLogo.png}
\end{titlepage}
%%%%%%%%%%%%%%%%%%%%%%%%%%%%%%%%%%%%%%%%%%%%%%%%%%%%%%%%%%%%%%%%%%%%%%%%%%
%%%%%%%%%%%%%%%
\tableofcontents
\pagebreak
%%%%%%%%%%%%%%%%%%%%%%%%%%%%%%%%%%%%%%%%%%%%%%%%%%%%%%%%%%%%%%%%%%%%%%%%%%
%%%%%%%%%%%%%%%
\renewcommand{\thesection}{\arabic{section}}
\section{Introduction}
The purpose of this lab is to become familiar with using Fast Fourier
Transforms. Along with becoming familiar with Fast Fourier Transforms using
Python, we will also be introduced to cleaning up these bode plots as well.
\section{Equations}
The equations shown down below are the expressions that were tested by the
Fast Fourier Transform function. The equation used for each task is 
represented by its respected task number.
\begin{equation}
    x_{1} = cos(2\pi t)
\end{equation}
\begin{equation}
    x_{2} = 5sin(2\pi t)
\end{equation}
\begin{equation}
    x_{3} = 2cos((2\pi \cdot 2t) - 2) + sin^{2}((2\pi \cdot 6t) + 3)
\end{equation}

\section{Methodology}
\subsection{Task 1}
To start, I first needed to create the user defined code for the Fast
Fourier Transform function. Most of the code was provided for this one.
I then needed to plot equation 1. I found the variables for the magnitude,
frequency, and angle of equation 1 using the Fast Fourier Transform
function with equation 1 as the input. The code for the user defined
function and the the first plot can be seen down below.
\begin{lstlisting}[language=Python]
steps = 1e-3
t = np.arange(1e-5, 2, steps)
fs = 100

def fastft(x):
    N = len( x ) # find the length of the signal
    X_fft = scipy.fftpack.fft(x) # perform the fast Fourier transform (fft)
    X_fft_shifted = scipy.fftpack.fftshift(X_fft) 
        # shift zero frequency components
        # to the center of the spectrum
    freq = np.arange (-N/2 , N/2) * fs / N 
        # compute the frequencies for the output
        # signal , (fs is the sampling frequency a needs to be defined previously 
        #  in your code
    X_mag = np.abs( X_fft_shifted ) / N # compute the magnitudes of the signal
    X_phi = np.angle( X_fft_shifted ) # compute the phases of the signal
# ----- End of user defined function ----- #    
    return freq, X_mag, X_phi

# Task 1
x1 = np.cos(2*np.pi*t)
freq1, X_mag1, X_phi1 = fastft(x1)


plt.figure(figsize=(12,8))
plt.subplot(3,2,(1,2))
plt.plot(t,x1)
plt.title('Task 1 - User Definited FFT of x1(t)')
plt.ylabel('x(t)')
plt.xlabel('t[s]')
plt.grid(True)

plt.subplot(3,2,3)
plt.stem(freq1, X_mag1)
plt.ylabel('|x(t)|') 
plt.grid(which='both')

plt.subplot(3,2,4)
plt.stem(freq1, X_mag1) 
plt.xlim(-2,2)
plt.grid(which='both')

plt.subplot(3,2,5)
plt.stem(freq1, X_phi1)
plt.xlabel('f[Hz]') 
plt.ylabel('/_x(t)') 

plt.subplot(3,2,6)
plt.stem(freq1, X_phi1)
plt.xlim(-2,2)
plt.xlabel('f[Hz]') 
plt.ylabel('/_x(t)') 
plt.show()
\end{lstlisting}
\newpage
\subsection{Tasks 2 \& 3}
For this section I did the exact same thing from task 1, except I used equations
2 and 3 instead.
\begin{lstlisting}[language=Python]
# Task 2
x2 = 5*np.sin(2*np.pi*t)
freq2, X_mag2, X_phi2 = fastft(x2)

plt.figure(figsize=(12,8))
plt.subplot(3,2,(1,2))
plt.plot(t,x2)
plt.title('Task 2 - User Definited FFT of x2(t)')
plt.ylabel('x(t)')
plt.xlabel('t[s]')
plt.grid(True)

plt.subplot(3,2,3)
plt.stem(freq2, X_mag2)
plt.ylabel('|x(t)|') 
plt.grid(which='both')

plt.subplot(3,2,4)
plt.stem(freq2, X_mag2) 
plt.xlim(-2,2)
plt.grid(which='both')

plt.subplot(3,2,5)
plt.stem(freq2, X_phi2)
plt.xlabel('f[Hz]') 
plt.ylabel('/_x(t)') 

plt.subplot(3,2,6)
plt.stem(freq2, X_phi2)
plt.xlim(-2,2)
plt.xlabel('f[Hz]') 
plt.ylabel('/_x(t)') 
plt.show()

# Task 3
x3 = 2*np.cos((2*np.pi*6*t) - 2) + (np.sin((2*np.pi*6*t) + 3) ** 2) 
freq3, X_mag3, X_phi3 = fastft(x3)


plt.figure(figsize=(12,8))
plt.subplot(3,2,(1,2))
plt.plot(t,x3)
plt.title('Task 2 - User Definited FFT of x3(t)')
plt.ylabel('x(t)')
plt.xlabel('t[s]')
plt.grid(True)

plt.subplot(3,2,3)
plt.stem(freq3, X_mag3)
plt.ylabel('|x(t)|') 
plt.grid(which='both')

plt.subplot(3,2,4)
plt.stem(freq3, X_mag3) 
plt.xlim(-2,2)
plt.grid(which='both')

plt.subplot(3,2,5)
plt.stem(freq3, X_phi3)
plt.xlabel('f[Hz]') 
plt.ylabel('/_x(t)') 

plt.subplot(3,2,6)
plt.stem(freq3, X_phi3)
plt.xlim(-2,2)
plt.xlabel('f[Hz]') 
plt.ylabel('/_x(t)') 
plt.show()
\end{lstlisting}
\subsection{Task 4}
For the fourth task I needed to create a version of the Fast Fourier 
Transform function that cleans up all of the noise that was displayed from 
the last one. This was done by using a for loop and an if statement to see
at what points the magnitude is insignificant and then set the angle to 
zero. I then graphed the three functions again just like before except 
using the new clean function instead.

\begin{lstlisting}[language=Python]
# Task4
def cleanfft(x):
    N = len( x ) # find the length of the signal
    X_fft = scipy.fftpack.fft(x) # perform the fast Fourier transform (fft)
    X_fft_shifted = scipy.fftpack.fftshift(X_fft) # shift zero frequency components
                                                 # to the center of the spectrum
    freq = np.arange (-N/2 , N/2) * fs / N # compute the frequencies for the output
                                           # signal , (fs is the sampling frequency and
                                           # needs to be defined previously in your code
    X_mag = np.abs( X_fft_shifted ) / N # compute the magnitudes of the signal
    X_phi = np.angle( X_fft_shifted ) # compute the phases of the signal
    
    for i in range(len(X_phi)):
        if abs(X_mag[i]) < 1e-10:
            X_phi[i] = 0
# ----- End of user defined function ----- #    
    return freq, X_mag, X_phi
\end{lstlisting}

\subsection{Task 5}
For the last task, I tested the clean Fast Fourier Transform function on 
the square wave from lab 8. I did this by copying the ak, bk, and summation
functions from lab 8. I then inputted an N size of 15 into the square
function and then put that function into the clean Fast Fourier Transform
function.
\begin{lstlisting}[language=Python]
# Task 5
t = np.arange(0, 16, steps)
def ak(k):
    a = np.zeros(t.shape)
    return a

def bk(k):
    b = np.zeros(t.shape)
    b = 2/(k*np.pi) * (1 - np.cos(k*np.pi))
    return b

def x(t,k):
    T = 8
    for i in range(k):
        if i == 0:
            x = (1/2)*ak(i)
        else:
            x += ak(i) + bk(i)*np.sin((i*2*np.pi*t)/T)
    return x 

x15 = x(t,15)
freqc15, X_magc15, X_phic15 = cleanfft(x15)

plt.figure(figsize=(12,8))
plt.subplot(3,2,(1,2))
plt.plot(t,x15)
plt.title('Task 2 - Clean User Definited FFT of Square Wave from Lab 8')
plt.ylabel('x(t)')
plt.xlabel('t[s]')
plt.grid(True)

plt.subplot(3,2,3)
plt.stem(freqc15, X_magc15)
plt.ylabel('|x(t)|') 
plt.grid(which='both')

plt.subplot(3,2,4)
plt.stem(freqc15, X_magc15) 
plt.xlim(-2,2)
plt.grid(which='both')

plt.subplot(3,2,5)
plt.stem(freqc15, X_phic15)
plt.xlabel('f[Hz]') 
plt.ylabel('/_x(t)') 

plt.subplot(3,2,6)
plt.stem(freqc15, X_phic15)
plt.xlim(-2,2)
plt.xlabel('f[Hz]') 
plt.ylabel('/_x(t)') 
plt.show()
\end{lstlisting}

\section{Results}
\subsection{Part 1}
\begin{figure}[H]
\begin{center}
\caption{Task 1}
\includegraphics[scale=0.50]{task1.png}
\end{center}
\end{figure}

\begin{figure}[H]
\begin{center}
\caption{Task 2}
\includegraphics[scale=0.50]{task2.png}
\end{center}
\end{figure}

\begin{figure}[H]
\begin{center}
\caption{Task 3}
\includegraphics[scale=0.50]{task3.png}
\end{center}
\end{figure}

\begin{figure}[H]
\begin{center}
\caption{Task 1 Clean}
\includegraphics[scale=0.50]{task1clean.png}
\end{center}
\end{figure}

\begin{figure}[H]
\begin{center}
\caption{Task 2 Clean}
\includegraphics[scale=0.50]{task2clean.png}
\end{center}
\end{figure}

\begin{figure}[H]
\begin{center}
\caption{Task 3 Clean}
\includegraphics[scale=0.50]{task3clean.png}
\end{center}
\end{figure}

\begin{figure}[H]
\begin{center}
\caption{Task 5 Clean Square Wave}
\includegraphics[scale=0.50]{squareClean.png}
\end{center}
\end{figure}

\section{Error Analysis}
This lab went pretty smoothly with almost no errors. The only problem
I had was in the beginning when I thought I needed to create two
different timer periods for each of the different equations
so that I could account for the zoomed in plot. However,
I found out it would be easier just to use the normal time period and
then just limit the x value on the graph so that it zooms in.

\section{Questions}
1. What happens if fs is lower? If it is higher? fs in your report must span a few orders of magnitude.

When fs is lower the points for the magnitude and angle are much closer together,
so the plot is more narrow. When fs is higher the points are more spread out,
so the plot is wider. It is worth noting that it seems to take quite a bit
for this to happen. That is, I had to move by powers of 10 to make much of
a difference in how much it spread out.

2. What difference does eliminating the small phase magnitudes make?

It actually makes a humongous difference. At least in my graphs, it got
rid of a majority of the noise that was affecting the angle. Because of
this the angle is much clearer in the clean versions.

3. Verify your results from Tasks 1 and 2 using Fourier transforms of
cosine and sine. Explain why your results are correct. You will need the
transforms in terms of Hz, not rad/s. For example, the Fourier transform
of cosine (in Hz) is:
\begin{equation}
    F\{cos(2\pi f_{0}t)\} = \frac{1}{2}[\delta(f − f_{0}) + \delta(f + f_{0})]
\end{equation}
The Fourier transform of the cosine function can definitely be verified.
Using equation 4 shown above, it can be seen that the Fourier Transform
of the cosine function would have delta functions at 1 and -1 for the
magnitude. This is exactly what is shown on either of the graphs made
for task one. As for the sine function it is very similar to the Fourier
Transform shown in equation 4. The differences are that it has a j in
front and a subtraction sign instead of an addition sign. This difference
is also shown in the graph since the phases switch which one is pointing
up and which one is pointing down.

4. Leave any feedback on the clarity of lab tasks, expectations, and
deliverables.

I think for this lab I was a little confused, starting out, on how the
tasks should be completed. However, I was able to figure out what I was
doing wrong.

\section{Conclusion}
I think this lab was very useful in learning about how to perform
Fourier series using Python. I also think it was helpful to know that a
plot can be cleaned up really smoothly if we only focus on the important
parts. I thought it was also really helpful to compare the results of the
zoomed in plots to the actual calculated Fourier Series Transforms. It
allowed me to get a better idea of how to read those kinds of plots.

\end{document}
%This template was created by Roza Aceska.