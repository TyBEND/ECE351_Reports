%%%%%%%%%%%%%%%%%%%%%%%%%%%%%%%%%%%%%%%%%%%%%%%%%%%%%%%%%%%%%%%%
%                                                              %
% Tyler Bendele                                                %
% ECE351 and Section 51                                        %
% Lab 5                                                        %
% Due Date: February 22, 2022                                  %
% In this lab we find the step and impulse response of an      %
% RLC Bandpass Filter                                          %
%                                                              %
%%%%%%%%%%%%%%%%%%%%%%%%%%%%%%%%%%%%%%%%%%%%%%%%%%%%%%%%%%%%%%%%
%%% DOCUMENT PREAMBLE %%%
\documentclass[12pt]{report}
\usepackage[english]{babel}
%\usepackage{natbib}
\usepackage{url}
\usepackage[utf8x]{inputenc}
\usepackage{amsmath}
\usepackage{graphicx}
\graphicspath{{images/}}
\usepackage{parskip}
\usepackage{fancyhdr}
\usepackage{vmargin}
\usepackage{listings}
\usepackage{hyperref}
\usepackage{xcolor}
\usepackage{float}
\definecolor{codegreen}{rgb}{0,0.6,0}
\definecolor{codegray}{rgb}{0.5,0.5,0.5}
\definecolor{codeblue}{rgb}{0,0,0.95}
\definecolor{backcolour}{rgb}{0.95,0.95,0.92}
\lstdefinestyle{mystyle}{
backgroundcolor=\color{backcolour},
commentstyle=\color{codegreen},
keywordstyle=\color{codeblue},
numberstyle=\tiny\color{codegray},
stringstyle=\color{codegreen},
basicstyle=\ttfamily\footnotesize,
breakatwhitespace=false,
breaklines=true,
captionpos=b,
keepspaces=true,
numbers=left,
numbersep=5pt,
showspaces=false,
showstringspaces=false,
showtabs=false,
tabsize=2
}
\lstset{style=mystyle}
\setmarginsrb{3 cm}{2.5 cm}{3 cm}{2.5 cm}{1 cm}{1.5 cm}{1 cm}{1.5 cm}
\title{Lab 5: Step and Impulse Response of a RLC Band Pass Filter}
% Title
\author{Tyler Bendele}
% Author
\date{February 22, 2022}
% Date
\makeatletter
\let\thetitle\@title
\let\theauthor\@author
\let\thedate\@date
\makeatother
\pagestyle{fancy}
\fancyhf{}
\rhead{\theauthor}
\lhead{\thetitle}
\cfoot{\thepage}
%%%%%%%%%%%%%%%%%%%%%%%%%%%%%%%%%%%%%%%%%%%%
\begin{document}
%%%%%%%%%%%%%%%%%%%%%%%%%%%%%%%%%%%%%%%%%%%%%%%%%%%%%%%%%%%%%%%%%%%%%%%%%%
%%%%%%%%%%%%%%%
\begin{titlepage}
\centering
\vspace*{0.5 cm}
% \includegraphics[scale = 0.075]{bsulogo.png}\\[1.0 cm] % 

\begin{center}    \textsc{\Large   ECE 351 - Section \#51 }\\[2.0 cm]
\end{center}% University Name
\textsc{\Large Signals and Systems  }\\[0.5 cm] % Course 

\rule{\linewidth}{0.2 mm} \\[0.4 cm]
{ \huge \bfseries \thetitle}\\
\rule{\linewidth}{0.2 mm} \\[1.5 cm]
\begin{minipage}{0.4\textwidth}
\begin{flushleft} \large
% \emph{Submitted To:}\\
% Name\\
% Affiliation\\
%contact info\\
\end{flushleft}
\end{minipage}~
\begin{minipage}{0.4\textwidth}
\begin{flushright} \large
\emph{Submitted By :} \\
Tyler Bendele
\end{flushright}
\end{minipage}\\[2 cm]
% \includegraphics[scale = 0.5]{PICMathLogo.png}
\end{titlepage}
%%%%%%%%%%%%%%%%%%%%%%%%%%%%%%%%%%%%%%%%%%%%%%%%%%%%%%%%%%%%%%%%%%%%%%%%%%
%%%%%%%%%%%%%%%
\tableofcontents
\pagebreak
%%%%%%%%%%%%%%%%%%%%%%%%%%%%%%%%%%%%%%%%%%%%%%%%%%%%%%%%%%%%%%%%%%%%%%%%%%
%%%%%%%%%%%%%%%
\renewcommand{\thesection}{\arabic{section}}
\section{Introduction}
In this lab the goal is to find the time-domain response of a RLC Band Pass
filter to the impulse and step inputs. This was done by first finding
the transfer function and impulse response of the RLC Band Pass Filter
provided in the pre-lab. Then through testing it in Python.
\section{Equations}
The equations shown below are the transfer function found from the RLC
Band Pass Filter, the impulse Response found, and the Final Value Theorem.
The Final Value Theorem is used to find the impulse as a function goes off
to infinity. It only works on stable functions, that is transfer functions
with roots that correspond to the left side of the y-axis.
\begin{equation}
    H(s) = \frac{\frac{s}{RC}}{s^{2} + \frac{s}{RC} + \frac{1}{LC}}
\end{equation}
\begin{equation}
    y(t) = ((-10355.6)e^{-5000t} + sin(18584t + 105.06)) \cdot u(t)
\end{equation}
\begin{equation}
    \lim_{t\to \infty}\{f(t)\} = \lim_{s\to 0}\{sF(s)\}
\end{equation}

\section{Methodology}
\subsection{Part 1}
Much of the work done for the lab, was actually done in the pre-lab, where
we had to find the Transfer function and Impulse response of a given RLC
Band Pass Filter. These equations that I ended up using are shown in 
equations 1 and 2. To start I first added in the user defined functions: 
the step response and the impulse response (the function shown in equation
2). 

\begin{lstlisting}[language=Python]
steps = 1e-6
t = np.arange(0, 1.2e-3+steps, steps) 

def u(t):          #step function
    y = np.zeros(t.shape)
    for i in range(len(t)):
        if t[i] > 0:
            y[i] = 1
        else:
            y[i] = 0
    return y

def impulse(t):
    y = -10355.6 * np.exp(-5000*t) * np.sin(18584*t + 105.06) * u(t)
    return y
\end{lstlisting}
Next, I needed to compare my hand calculated impulse equation to the 
one found from the impulse function using the Scipy library. I did this by
entering the first equation in the following format and then graphing them
on the same figure.
\begin{lstlisting}[language=Python]
um = [0, 10000, 0] # Creates a matrix for the numerator
den = [1, 10000, 3.8037e8] # Creates a matrix for the denominator

tout , yout = sig.impulse(( num , den) , T = t )

# Plot tout , yout
plt.figure(figsize = (10, 7))
plt.subplot(2, 1, 1)
plt.plot(t, impulse(t))
plt.grid()
plt.ylabel('Hand Calculated')
plt.title('Impulse Response of a RLC Band Pass Filter')

plt.subplot(2, 1, 2)
plt.plot(tout, yout)
plt.grid()
plt.ylabel('Scipy Impulse Solution') 
plt.xlabel('t')
plt.show()
\end{lstlisting}
\subsection{Part 2}
For Part 2 task 1, we had to find the step response using the Scipy
step function. The code for this can be seen down below. I then had
to find the step response in the Laplace domain of the Transfer function 
using the Final Value Theorem for the second task in part 2. 
\begin{lstlisting}[language=Python]
tout2 , yout2 = sig.step((num, den), T=t)

plt.figure(figsize = (10, 7))
plt.subplot(1, 1, 1)
plt.plot(tout2, yout2)
plt.grid()
plt.ylabel('Scipy Step Response') 
plt.xlabel('t')
plt.show()
\end{lstlisting}

\section{Results}

\subsection{Part 1}
\begin{figure}[H]
\begin{center}
\caption{Part 1, Tasks 1 \& 2, Impulse Response}
\includegraphics[scale=0.50]{part1pics.png}
\end{center}
\end{figure}

\subsection{Part 2}
\begin{figure}[H]
\begin{center}
\caption{Part 2, Task 1, Step Response}
\includegraphics[scale=0.50]{part2pics.png}
\end{center}
\end{figure}

\subsection{Part 2 Task 2 \& Task 3}
When applying the Final Value Theorem to the transfer function I found
that it also shows the step function going to zero. This shows that the function is
stable. Comparing this result to the graphs from part 1, it does confirm
that it is going to zero as t goes to infinity.

\section{Error Analysis}
This time around I don't think I had any errors. I did try doing the lab,
using more functions in order to do it completely using variables, but it 
seemed to be pretty difficult doing it that way.

\section{Questions}
1. Explain the result of the Final Value Theorem from Part 2 Task 2 in
terms of the physical circuit components.

Since the bottom part of the transfer function had negative roots, meaning
both discontinuities were on the left side of the y-axis, the Final Value
Theorem was a valid operation of the function. This allowed the result to
give a zero, showing that the function was stable. I want to say that this result from the
Final Value Theorem means that capacitance and
inductance were balanced. If there was too 
much capacitance, it would caused the filter
to be unstable.

2. Leave any feedback on the clarity of the expectations, instructions, and deliverables.

It seems like everything in this lab was pretty clear on the clarity of the
expectations, instructions, and deliverables. 

\section{Conclusion}
This lab was pretty cool because I was able to find the impulse response
through several calculations and then compare it to the Scipy impulse
response function. They both ended up being identical. It was also interesting to learn
more about how the Final Value Theorem works. I learned it is useful in determining
how stable a function is and if it is stable then most of the time it we also know it
goes to zero as it goes to infinity. I don't believe I have any recommendations for
future labs. 

\end{document}
%This template was created by Roza Aceska.